\section{Direkte Verfahren}
Direkte Verfahren Lösen ein Problem nach endlich vielen Schritten. Verwendung: kleine, vollbesetzte Matrizen.
\subsection{LR-Zerlegung}
\subsubsection{Ziel}
\begin{align*}
	A &= LR \\
	\begin{pmatrix}
		a_{11} & \dots & a_{1n} \\
		\vdots & \ddots & \vdots \\
		a_{n1} & \dots & a_{nn}
	\end{pmatrix} &= \begin{pmatrix}
		1 & 0 & \dots & 0 \\
		* & 1 \\
		\vdots &  & \ddots & \vdots\\
		* & & \dots & 1
	\end{pmatrix} \begin{pmatrix}
		* & * & \dots & * \\
		0 & * \\
		\vdots &  & \ddots & \vdots\\
		0 & & \dots & *
	\end{pmatrix}
\end{align*}
\subsubsection{Algorithmus}
\begin{enumerate}
	\item $i$-te Zeile in R übertragen
	\item $i$-te Spalte dividiert durch $a_{ii}$ in L über nehmen. Erstes Element der Spalte gleich 1 setzten 
	\item Mit $i$-ter Zeile die $i$-te Spalte eliminieren 
\end{enumerate}
\subsubsection{Komplexität}
$\BigO(n^3)$
\subsubsection{Anwendung}
\begin{itemize}
	\item $\det(A)=\det(L)\times\det(R)=1\times\det(R)$
	\item Lösen mehrerer GLS:
		\begin{itemize}
			\item $Ly=b$ mit Vorwärtssubstitution $\BigO(n^2)$
			\item $Rx=y$ mit Rückwärtssubstitution $\BigO(n^2)$
		\end{itemize}
\end{itemize}
%TODO: PLR-Zerlegung
\subsubsection{LRP-Zerlegung}
\begin{align*}
	A &= PLR
\end{align*}
\subsection{QR-Zerlegung}
\subsubsection{Ziel}
\begin{align*}
	A = QR
\end{align*}
\subsubsection{Housholder-Spiegelungen}
%TODO: Householder-Spiegelungen
Mit einer Housholder-Spiegelung in eriner Spalte Nullen einfügen (außer Diagonalelement)

$\to$ nach $n-1$ Schritten erhält man die Dreiecksmatrix $R$

\begin{align*}
	R &= H_{n-1}\dots H_2H_1A \\
	Q &= (H_{n-1}\dots H_2H_1)^{-1} = H_1 H_2 \dots H_{n-1}
\end{align*}
\subsubsection{Givens-Rotationen}
Mit einer Givens-Rotation ein Element (unterhalb der Diagonalen) zu Null machen

$\to$ nach $n(n-1)/2$ Schritten erhält man die Dreiecksmatrix $R$

\begin{align*}
	J_{ij}(\varphi)=\begin{pmatrix}
		1 \\
		& \ddots \\
		& & c & &-s \\
		& &  &\ddots \\
		& & s & & c \\
		& & & & & \ddots \\
		& & & & & & 1
	\end{pmatrix}
\end{align*}
Wobei $c_1$ an Position $jj$ ist und $c_2$ an Position $ii$
\begin{align*}
	c &= \cos(\varphi) = \frac{\sigma\cdot a_{jj}}{\sqrt{a^2_{jj}+a^2_{ij}}} \\
	s &= \sin(\varphi)=\frac{-\sigma\cdot a_{ij}}{\sqrt{a^2_{jj}+a^2_{ij}}} \\
	\sigma &= \text{sign}(a_{jj})
\end{align*}
Ergebnis:
\begin{align*}
	R &= J_{m,n^*}\dots J_{2,1}A \\
	Q &= J_{2,1}^T\dots J_{m,n^*}^T \\
	n^* &= \min\{m-1, n\}
\end{align*}
\subsection{Cholesky-Zerlegung}
Wenn $A$ symmetrisch und positiv definit ist kann man A faktorisieren in 
$$
	A=LDL^T
$$
Wobei $L$ das $L$ der LR-Zerlegung ist und $D$ der Diagonalanteil von $R$











