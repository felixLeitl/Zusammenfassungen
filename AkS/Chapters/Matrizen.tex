\section{Matrizen}
\subsection{Orthogonal}
Eine Matrix ist orthogonal, falls eine der Bedingungen erfüllt ist:
\begin{itemize}
	\item $Q^TQ=Id$
	\item $QQ^T=Id$
	\item Spalten oder Zeilen bilden eine Orthonomalbasis
	\item Die Abbildung $Q$ ist winkel- und längentreu
	\item $Q$ erhält das Skalarprdukt: $Qx\circ Qy = x \circ y$
\end{itemize}
\subsection{Skalarprodukt}
$x\circ y=\displaystyle\Sigma_{i=1}^nx_iy_i$
\subsection{Tridiagonalmatrix}
Die inverse einer tridiagonalen Matrix ist in der Regel voll besetzt