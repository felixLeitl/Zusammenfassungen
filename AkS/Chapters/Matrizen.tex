\section{Matrizen}
\subsection{Orthogonal}
Eine Matrix ist orthogonal, falls eine der Bedingungen erfüllt ist:
\begin{itemize}
	\item $Q^TQ=Id$
	\item $QQ^T=Id$
	\item Spalten oder Zeilen bilden eine Orthonomalbasis
	\item Die Abbildung $Q$ ist winkel- und längentreu
	\item $Q$ erhält das Skalarprdukt: $Qx\circ Qy = x \circ y$
\end{itemize}
\subsection{Skalarprodukt}
$x\circ y=\displaystyle\Sigma_{i=1}^nx_iy_i$
\subsection{Tridiagonalmatrix}
Die inverse einer tridiagonalen Matrix ist in der Regel voll besetzt
\subsection{Normen}
Eigenschaften:
\begin{itemize}
	\item definit: $x\not= 0 \Rarr ||x|| > 0$
	\item homogen: $||\lambda x|| = |\lambda|\cdot||x||$
	\item sub-additiv: $||x+y|| \leq ||x|| + ||y||$
\end{itemize}
\subsubsection{Matrix-Norm bzw. Operator-Norm}
Erfüllt Normeigenschaften und mehr:
\begin{itemize}
	\item $|||Id|||=1$
	\item sub-multiplikativ: $|||AB|||\leq|||A|||\cdot|||B|||$
	\item mit der Vektornorm kompatibel: $||Ax|| \leq |||A|||\cdot||x||$ 
	\item $|||A|||\geq |\lambda|$
\end{itemize}
Beispiele:
\begin{itemize}
	\item Spalten-Summen-Norm: $|||A|||_1$
		$$
			|||A|||_1 = \max_j \{\Sigma_i|a_{ij}|\}
		$$
	\item Zeilen-Summen-Norm: $|||A|||_\infty$
		$$
			|||A|||_\infty = \max_i \{\Sigma_j|a_{ij}|\}
		$$
	\item Spektral-Norm: $|||A|||_2$
		$$
			|||A|||_2 = \sqrt{\lambda\max(A^TA)}
		$$
	\item Frobenius-Norm: $||A||_F$
		$$
			||A||_F = \sqrt{\Sigma_{i=1}^m\Sigma_{j=1}^na_{ij}^2}
		$$
\end{itemize}
\subsubsection{Konditionszahl}
$$
	\kappa(A) = \frac{\max_{x\in\R^n, ||x||=1} ||Ax||}{\min_{x\in\R^n, ||x||=1} ||Ax||}
$$
\subsection{Spektralsatz}
Es sei $A\in\R^{m\times m}$ eine reelle symmetrische Matrix. Dann gibt es eine Orthonomalbasis aus Eigenvektoren bzw. $A=VDV^T$, wobei $D$ die Diagonalmatrix aller EW ist und die Spalten von $V$ die normierten EV sind.