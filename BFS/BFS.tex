\documentclass{article}
\usepackage{femape}
\usepackage[a4paper, total={6in, 8in}]{geometry}
\title{Berechenbarkeit und formale Sprachen}
\author{Felix Leitl}
\begin{document}
\maketitle
\tableofcontents
\newpage
\section{Turingmaschine}
\subsection{1-Band TM}
Turingmaschine $M=(Q, \Sigma, \Gamma, \delta, q_0, F)$:
\begin{itemize}
	\item $Q:$ endliche Zustandsmenge
	\item $\Sigma:$ endliches Eingabealphabet
	\item $\Gamma:$ endliches Bandalphabet $\Sigma \subsetneq \Gamma$
	\item $B:$ Blank, $B\in\Gamma, B\notin\Sigma$
	\item $q_0:$ $q_0\in Q$ Startzustand
	\item $F:$ akzeptierende Endzustände, $F\subseteq Q$
	\item das Programm $\delta:$ $Q\times\Gamma\to Q\times\Gamma\times\{R,L,N\}$ eine partielle Funktion, wobei es für Endzustände keine Übergänge geben soll
	\item Zu Beginn steht der Lese-/Schreibkopf auf dem ersten Zeichen der Eingabe
	\item Eingabe: $w=w_1w_2\dots w_n\in\Sigma^*$
	\item $\epsilon:$ leeres Wort
	\item $L\subseteq\Sigma^*$ ist Sprache über dem Alphabet $\Sigma$
\end{itemize}
\subsection{Deltatabelle}
$Q=\{q_0, q_1\}, \Sigma = \{0, 1\}, \Gamma \{0, 1, B\}, F=\{q_1\}$ \newline \newline
\begin{tabular}{c|c|c|c}
	$\delta$ & 0 & 1 & $B$ \\
	\hline
	$q_0$ & $(q_0, 0, R)$ & $(q_1, 1, R)$ & - \\
	\hline
	$q_1$ & - & - & - 
\end{tabular}
\subsection{Konfiguration}
TM $M$ ist in Konfiguration $K=\alpha q \beta$ $(\Gamma^*\times Q \times \Gamma^*)$, wobei der Schreib-/Lesekopf auf dem ersten Zeichen von $\beta$ steht. \newline
Eine direkte Nachfolgekonfiguration von $\alpha q\beta$ ist: $\alpha q \beta \vdash \alpha'q'\beta'$ \newline
$i-$te Nachfolgekonfiguration $\alpha q \beta \vdash K_1 \vdash \dots \vdash K_{i-1} \vdash \alpha'q'\beta'=\alpha q\beta \vdash^i\alpha q\beta$ \newline
Nachfolgekonfiguration: $\alpha q \beta \vdash^* \alpha'q'\beta'$
\subsection{Begriffe}
\begin{itemize}
	\item akzeptieren: Falls es $\alpha, \beta \in \Gamma^*$ und $q\in F$ gibt mit $q_0x\vdash^*\alpha q \beta$
	\item L(M): Menge aller von M akzeptierter Eingaben $x\in\Sigma^*$
	\item entscheidet: M hält mit Eingabe $x\in \Sigma^*$ nach endlich vielen Schritten 
	\item rekursiv aufzählbar: 
		\begin{itemize}
			\item $L\subseteq \Sigma^*$ ist rekursiv aufzählbar, wenn es eine TM $M$ gibt mit $L(M)=L$
			\item es gibt eine surjektive Funktion $g:{0, 1}^*\to L$
		\end{itemize}
	\item entscheidbar/rekursiv: 
		\begin{itemize} 
			\item wenn es eine deterministische 1-Band-TM $M$ gibt, die L entscheidet
			\item $L$ und $\overline{L}$ sind rekursiv aufzählbar
		\end{itemize}
\end{itemize}
\subsection{Programmiertechniken}
\subsubsection{Endlicher Speicher}
Man merkt sich die Zeichen im Zustand \newline
$\Gamma=\Sigma\cup\{B\}, Q=(\{q_0\}\times\Sigma)\cup\{q_0, q_1\}, \text{ Startzustand } q_0, F=\{q_1\}$
\subsubsection{Unterprogramme}
Wenn man eine TM \glqq{}programmiert\grqq{}, kann man sagen: Man benutzt ein Unterprogramm um eine bestimmte Aufgabe zu lösen
\subsubsection{Spurtechnik}
\begin{tabular}{cc|c|c|c|cc}
	\hline
	&&U&N&I&&\\
	\hline
	&&E&R&L&&\\
	\hline
	&&N&B&G&&\\
	\hline
\end{tabular} \newline
Das erste Zeichen wäre $\begin{pmatrix}
	U\\
	E\\
	N
\end{pmatrix}$
\subsection{Gödelnummer}
$\langle M\rangle$ ist die Gödelnummer (Bauplan von M). Sie ist die Repräsentation der TM $M$ als natürliche Zahl
\subsubsection{Universelle TM}
Eine TM $\tilde{M}$ hießt universell, wenn sie sich mit der Eingabe $\langle M \rangle x, x\in\{0, 1\}^*$ so verhält, wie $M$ gestartet mit $x$
\subsection{Halteproblem}
\subsubsection{Allgemeines Halteproblem}
$$
	H=\{\langle M\rangle|M \text{ ist deterministische 1-Band-TM, die, gestartet mit Eingabe } w, \text{ hält} \}
$$
\subsubsection{Initiales Halteproblem}
$$
	H_\epsilon=\{\langle M\rangle | M \text{ ist deterministische 1-Band-TM, die, gestartet mit Eingabe }\epsilon, \text{hält}\}
$$
\subsection{Reduktion}
\begin{itemize}
	\item Eine Funktion ist berechenbar, wenn es eine TM $M_f$ gibt, für die mit $x\in\{0,1\}^*$ gilt:
		\begin{itemize}
			\item Ist $f(x)$ definiert, so hält $M_f$ mit der Eingabe $x$ und $f(x)$ steht auf dem Band
			\item Ist $f(x)$ undefiniert, so hält $M_f$ gestartet mit $x$ nicht
		\end{itemize}
	\item Eine Funktion ist total, wenn alle $x\in\{0,1\}^*$ definiert und berechenbar sind 
\end{itemize}
Eine Reduktion ist eine total berechenbare Funktion $f:\{0,1\}^*\to\{0,1^*\}$, für die gilt:
$$
	x\in L_1 \Leftrightarrow f(x)\in L_2
$$ 
Wir schreiben \glqq{}$L_1\leq L_2$\grqq{} und sagen \glqq{} $L_1$ wird auf $L_2$ reduziert\glqq{}
\subsubsection{$L_1\leq L_2$}
\begin{itemize}
	\item $L_2$ entscheidbar \Rightarrow $L_1$ entscheidbar
	\item $L_2$ rekursiv aufzählbar \Rightarrow $L_1$ rekursiv aufzählbar
	\item $L_1$ unentscheidbar \Rightarrow $L_2$ unentscheidbar
	\item $L_1$ nicht rekursiv aufzählbar \Rightarrow $L_2$ nicht rekursiv aufzählbar
\end{itemize}
\section{Nichtdeterministische Turingmaschine}
Nichtdeterministische TM $M=\{Q, \Sigma, \Gamma, \delta, q_0, F\}$, wobei nur $\delta$ anderes ist, als bei einer deterministischen TM. $\delta: Q\times \Gamma \to P(Q\times, \Gamma \times \{R, N, L\})$
\subsection{Begriffe}
\begin{itemize}
	\item akzeptieren: wenn es eine Rechnung $q_0x\vdash^*\alpha q\beta$ gibt
\end{itemize}
\subsection{P-NP}
\begin{itemize}
	\item $\text{DTIME}(t(n)):=\{L|\text{ Es gibt eine deterministische }\BigO(t(n)) \text{-zeitbeschränkte TM, die } L \text{ entscheidet}\}$
	\item $\text{NTIME}(t(n)):=\{L|\text{ Es gibt eine nichtdeterministische }\BigO(t(n)) \text{-zeitbeschränkte TM, die } L \text{ akzeptiert}\}$
\end{itemize}
\subsection{Sprachprobleme}
\subsubsection{Satisfiability Problem}
\begin{align*}
	\text{SAT}:=\{\langle\Phi\rangle|\Phi\text{ ist eine erfüllbare }KNF\}
\end{align*}
\subsubsection{Clique}
\begin{align*}
	\text{CLIQUE}:=\{&\langle G, k \rangle|k\in \N, G \text{ ist ein ungerichteter Graph} \\
	&\text{der einen vollständigen Teilgraphen der Größe } k \text{ enthält}\}
\end{align*}
\subsubsection{Independent-Set}
\begin{align*}
	\text{IS}:=\{&\langle G, k \rangle|k\in \N, G = (V,E) \text{ ist ungerichteter Graph}, \\
	&\exists U\subseteq V, |U|=k:\forall u, v\in U:\{u, v\}\notin E\}
\end{align*}
\subsubsection{Coloring}
\begin{align*}
	\text{COL}&:=\{\langle G, k\rangle|G \text{ ist ein ungerichteter Graph und } G \text{ ist } k\text{-färbbar}\} \\
	\text{3COL}&:=\{\langle G \rangle|G \text{ ist ein ungerichteter Graph und } G \text{ ist 3-färbbar}\}
\end{align*}
\subsubsection{Traveling-Salesman}
\begin{align*}
	\text{TSP}:=\{&\langle G, c, k\rangle|\text{ der Graph } G \text{ mit Kantengewicht } c:E\to\R \\
		& \text{enthält eine Rundreise mit Gewicht} \leq k\}
\end{align*}
\subsubsection{Vertex-Cover}
\begin{align*}
	\text{VC}:=\{&\langle G, k\rangle|k\in\N, G \text{ ist ein ungerichteter Graph} \\
	&\text{und hat eine Knotenüberdeckung der Größe } k\}
\end{align*}
\subsubsection{Binary-Programming}
\begin{align*}
	\text{BP}:=\{&\langle A, \vec{b}\rangle|A \text{ ist eine } m\times n \text{-Matrix mit ganzzahligen Einträgen, }\vec{b} \text{ ist ein Vektor mit } m \\
	&\text{ganzzahligen Einträgen und es gibt einen 0-1-Vektor } \vec{x}\in\{0,1\}^n \text{ mit } A\cdot\vec{x}\leq\vec{b}\}
\end{align*}
\subsection{Verifizierer}
Eine deterministische Turingmaschine $V_L$ heißt $t(n)$-beschränkter Verifizierer für $L$, wenn gilt:
\begin{enumerate}
	\item Die Eingaben von $V_L$ sind von der Form $x\#w, w, x\in\{0,1\}^*$
	\item Die Laufzeit ist in $\BigO(t(|x|))$
	\item Für alle $x\in\{0,1\}^*$:
		$$
			x\in L\Leftrightarrow\exists w:|w|\leq t(|x|) \text{ und } V_L \text{ akzeptiert } x\#w
		$$
		Dieses heißt Zertifikat von w
\end{enumerate}
$L\in NTIME(t(n))\Leftrightarrow\text{ es gibt einen } t(n)\text{-beschränkten Verifizierer } V_L \text{ für } L$ \newline
$NP=\{L|\text{ es gibt einen polynomiellen Verifizierer für } L \}$
\subsection{NP-Vollständig}
\subsubsection{Polynomiell reduzierbar}
Seien $L_1\subseteq\Sigma_1^*,L_2\subseteq\Sigma_2^*$. \newline
$L_1$ ist genau dann polynomiell reduzierbar auf $L_2(L_1\leq_pL_2)$, wenn gilt: 
\begin{enumerate}
	\item $L_1\leq L_2$ mittels Reduktionsfunktion $f$
	\item Die Laufzeit zur Berechnung von $f(x)$ ist $\BigO(|x|^k)$ für $k\in\N$
\end{enumerate}
\subsubsection{NP-schwer}
$$
	\forall L'\in NP:L'\leq_pL
$$
\subsubsection{NP-vollständig}
\begin{enumerate}
	\item $L\in NP$
	\item $L$ ist $NP$-schwer
\end{enumerate}
\section{Formale Sprachen}
Eine Grammatik $G$ vom Typ Chomsky-0 ist beschrieben durch ein 4-Tupel$(V,\Sigma,P,S)$ mit:
\begin{itemize}
	\item $V$: Endliche Menge an Variablen
	\item $\Sigma$: Endliche Menge der Terminalsymbole
	\item $S:S\in V$: Startsymbol
	\item $P\subseteq((V\cup\Sigma)^+\backslash\Sigma^*)\times(V\cup\Sigma)^*$: Endliche Menge von Produktionen/Ableitungsregeln: $u\to v$
\end{itemize}
Die erzeugte Sprache:
$$
	L(G):=\{w|S\xrightarrow{*}w, w\in\Sigma^*\}
$$
\subsection{Chomsky-Grammatiken}
%TODO: Definitionen einfügen
\begin{tabular}{c|c|c|c}
	Sprachtyp & Erzeuger & Erkenner & Definition \\
	\hline
	rekursiv aufzählbar & Chomsky-0 & Turingmachinen & \\
	\hline
	kontextsensitiv & Chomsky-1 & Linear beschränkte Turningmaschinen & $|u|\leq|v|$ \\
	\hline
	kontextfrei & Chomsky-2 & Kellerautomaten & $u\in V \& |u|=1$\\
	\hline
	regulär & Chomsky-3 & Automaten & $u\in V \& v\in\{\epsilon\}\cup\Sigma$ / $u\in V \& v\in\Sigma\circ V$
\end{tabular}
\subsection{Endliche Automaten}
$A=(Q,\Sigma, \delta, q_0, F)$:
\begin{itemize}
	\item $Q$: Endliche Menge der Zustände
	\item $\Sigma$: Endliches Alphabet, $Q\cap\Sigma=\emptyset$
	\item $\delta$: $Q\times\Sigma\to Q$ Übergangsfunktion
	\item $q_0$: Startzustand
	\item $F$: $F\subseteq Q$, akzeptierende Endzustände
\end{itemize}
\begin{tabular}{c|c|c}
	$\delta$ & 0 & 1\\
	\hline
	$q_0$ & $q_0$ & $q_0$\\
	$q_1$ & $q_0$ & $q_1$ 
\end{tabular}
$\Rarr \delta(q_0, 001)= \delta(q_0, 01)=\delta(q_0, 1)=q_1$
\subsection{Reguläre Pump-Eigenschaft}
$\exists n_L\in\N\ \forall z\in L,|z|\geq n_L\ \exists u,v,w\in\Sigma^*:uvw=z$ und
\begin{itemize}
	\item $|uv|\leq n_L$
	\item $v\not= \epsilon$
	\item $\forall i\geq0:uv^iw\in L$
\end{itemize}
$\Rarr$ Die Sprache $L$ ist regulär
\subsection{Kontextfreie Pum-Eigenschaft}
$\exists n_L\in\N\ \forall z\in L,|z|\geq n_L\ \exists u,v,w,x,y\in\Sigma^*:uvwxy=z$ und
\begin{itemize}
	\item $|vwx|\leq n_L$
	\item $vx\not= \epsilon$
	\item $\forall i\geq0:uv^iwx^iy\in L$
\end{itemize}
$\Rarr$ Die Sprache $L$ ist kontextfrei
\end{document}



















