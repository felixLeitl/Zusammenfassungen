\section{Grundlagen}
\subsection{Modellierung}
Ein Modell ist ein zweckgerichtetes Abbild der Wirklichkeit

Zweck:
\begin{itemize}
	\item Spezifizieren
	\item Konstruieren
	\item Visualisieren
	\item Dokumnetieren
\end{itemize}

\subsection{Warum Datenbanken}
\begin{itemize}
	\item Große Software-Systeme
	\item Viele Anwendungen/Benutzer arbeiten mit den gleichen Daten
	\item Daten sollen auch nach Ende eines Programms verfügbar bleiben
	\item Daten sollen vor Verlust geschützt werden
	\item Daten sollen konsistent bleiben 
\end{itemize}

\subsection{Vorteile einer Datenbank}
\begin{itemize}
	\item Anwendungsneutralität
	\item Vermeidung redundanter Daten
	\item Zentrale Kontrolle der Datenintegrität
	\item Synchronisation im Mehrnutzerbetrieb
	\item Fehlertoleranz
	\item Perfomance
	\item Skalierbarkeit
	\item Verkürzte Entwicklungszeiten für Anwendungen
	\item Umsetzung von Standarts 
\end{itemize}

\subsection{Nachteile}
\begin{itemize}
	\item Hohe initiale Kosten 
	\item General purpose software
	\item Signifikanter Overhead
\end{itemize}

\subsection{Begriffe}
\subsubsection{Datenbank}
Eine Datenbank ist eine Sammlung zusammenhängender Daten.
\begin{itemize}
	\item repräsentiert einen Ausschnitt der realen Welt (Miniwelt)
	\item Logisch kohärente Sammlung von Daten
	\item Hat definierten Zweck
\end{itemize}
\subsubsection{Datenbank-Management-System}
Sammlung von Programmen zur Verwaltung einer Datenbank
\begin{itemize}
	\item Erzeugung von DB
	\item Wartung von DB
	\item Konsistenter Zugriff auf DB
\end{itemize}
\subsubsection{Datenbanksystem}
\begin{itemize}
	\item DB + DBMS
\end{itemize}
\subsubsection{Datenbankanwendung}
\begin{itemize}
	\item DBS + Anwendungsprogramme
\end{itemize}
\subsubsection{Datenmodell}
\begin{itemize}
	\item Strukturierungsvorschrift für Daten (z.B. Tabellenform)
\end{itemize}
\subsubsection{Datenbankschema}
\begin{itemize}
	\item Beschreibung einer konkreten Datenbank
\end{itemize}
\subsubsection{Nutzdaten}
\begin{itemize}
	\item Eigentliche Datenbank 
\end{itemize}
\subsubsection{Metadaten}
\begin{itemize}
	\item Struktur der DB
	\item Information über Speicherungsstrukturen
\end{itemize}
\subsubsection{Konzeptionelles Schema}
\begin{itemize}
	\item Beschreibt sämtliche Daten auf logischer Ebene 
	\item z.B. \verb|Patient (NR. Krankenkasse, Laborwerte)|
\end{itemize}
\subsubsection{Externes Schema}
\begin{itemize}
	\item Beschreibt den für die Anwendung relevanten Teil einer DB auf logischer Ebene
	\item z.B. für den Artzt: \verb|Patient (Nr., Laborwerte)| und für die Verwaltung: \verb|Patient (Nr., Krankenkasse)|
\end{itemize}
\subsubsection{Internes Schema}
\begin{itemize}
	\item Beschreibt die interne Speicherungsstrukturen einer Datenbank
	\item Unsichtbar für Anwendung
	\item z.B. Index über Attribut \verb|Nr.| von \verb|Patient|
\end{itemize}

\subsection{Phasen des Datenbankentwurfs}
\begin{itemize}
	\item Konzeptioneller Entwurf
		\begin{itemize}
			\item Abbildung auf Semantisches Datenmodell (z.B. E/R-Modell)
		\end{itemize}
	\item Logischer Entwurf
		\begin{itemize}
			\item Abbildung auf Datenmodell
		\end{itemize}
\end{itemize}
















