\section{Mapping}
\subsection{Abbildungskonzepte}
% I hate this shit
\begin{tabular}{c|c}
	$\bold{ER}$-$\bold{Modell}$ & $\bold{Relationenmodell}$ \\
	\hline
	Entity-Typ & \glqq{}Entity\grqq{}-Relation \\
	1:1- oder 1:N-Beziehungstyp & Fremdschlüssel oder \\
	M:N-Beziehungstyp & Beziehungstabelle mit 2 FS \\
	N-ärer Beziehungstyp & Beziehungstabelle mit N FS \\
	Einfaches Attribut & Attribut \\
	Zusammengesetztes Attribut & Menge von Attributen \\
	Mehrwertiges Attribut & \glqq{}Attribut\grqq{}-Relation mit FS \\
	Wertebereich & Wertebereich \\
	Schlüsselattribut & Schlüsselkandidat $\to$ Primärschlüssel
\end{tabular}
\subsection{Algorithmus}
\subsubsection{Reguläre Entity-Typen}
\begin{itemize}
	\item Erzeuge eine Relation R, die alle einfachen Attribute von E umfasst
		\begin{itemize}
			\item Bei zusammengesetzten Attributen nur Komponenten als eigenständige Attribute
		\end{itemize}
	\item Wähle aus Schlüsselkandidaten einen Primärschlüssel
		\begin{itemize}
			\item zusammengesetzt $\to$ Komponenten bilden zusammen den Primärschlüssel
			\item Jeder Schlüsselkandidat, außer PS wird UNIQUE \& NOT NULL
		\end{itemize}
\end{itemize}
\subsubsection{Schwache Entity-Typen}
\begin{itemize}
	\item Erzeuge eine Relation, die alle einfachen Attribute von W umfasst
	\item Füge als Fremdschlüssel alle PS-Attribute der Owner-Typen ein
	\item PS wird Kombination aller FSA, zusammen mit partiellem Schlüssel (falls vorhanden)
\end{itemize}
\subsubsection{M:N-Beziehungen}
\begin{itemize}
	\item Erzeuge Relation die alle einfachen Attribute von X umfasst
	\item FS $\to$ PSA der beidem Relationen
	\item PS ist Kombination der FSA
\end{itemize}
\subsubsection{N:1-Beziehungen}
\begin{itemize}
	\item identifiziere die Relation, die dem Entity-Typ E auf der N-Seite des Beziehungstyps entspricht
	\item Füge den PS des anderen ET als FS in R ein
	\item Füge alle einfachen Attribute des Beziehungstyps X als Attribute in R ein
\end{itemize}
\subsubsection{1:1-Beziehungen}
\begin{itemize}
	\item Identifiziere Relationen R \& S
	\item Nehme den PS von S bzw. R als FS von R bzw. S auf UNIQUE
	\item Füge alle einfachen Attribute in R bzw. S ein
\end{itemize}
\subsubsection{Mehrwertige Attribute}
\begin{itemize}
	\item Erzeuge Relation R mit folgenden Attributen:
		\begin{itemize}
			\item Ein Attribut A, dass dem abzubildenden Attribut A entspricht
			\item Den PS K der Relation S, die zu E gehört, als FS auf S
		\end{itemize}
	\item Der PS der Relation R ist die Kombination von A \& K
\end{itemize}
\subsubsection{Mehrstellige Beziehungen}
\begin{itemize}
	\item Erzeuge Relation R, die alle einfachen Attribute von B umfasst
	\item FS $\to$ PS aller Relationen
	\item PS $\to$ Kombination aller FS
\end{itemize}
\subsubsection{Generalisierung/Spezialisierung}
siehe VL
\subsubsection{Kategorien}
siehe VL
