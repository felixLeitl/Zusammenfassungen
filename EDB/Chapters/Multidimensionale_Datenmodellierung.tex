\section{Multidimensionale Datenmodellierung}
\subsection{OLTP vs. OLAP}
\subsubsection{OLTP}
Online Transaction Processing
\subsubsection{OLAP}
Online Analytical Processing
\subsection{Relationenmodell vs. Multidimensionales Datenmodell}
\subsubsection{Relationenmodell}
\begin{itemize}
	\item Einfach, wenige Modellierungskonstrukte
	\item Anwendungsneutral
	\item Keine \glqq{}eingebaute\grqq{} Anwendungssemantik
\end{itemize} 
$\Rightarrow$ Nützlich in beliebigen Domänen, manchmal etwas komplizierter in der Anwendung
\subsubsection{Multidimensionales Datenmodell}
\begin{itemize}
	\item Komplexer, mehr Modellierungskonstruke
	\item Speziell auf Anwendung zur Datenanalyse zugeschnitten
\end{itemize}
$\Rightarrow$ Nur nützlich für analytische Zwecke
\subsection{Charakterisierung der Datenanalyse}
\begin{itemize}
	\item Qualifizierende und Quantifizierende Daten
		\begin{itemize}
			\item Spezielle funktionale Abhängigkeiten $\Rightarrow$ spezifische Repräsentation
		\end{itemize}
	\item Klassifikationshierachien
		\begin{itemize}
			\item Aggregierende Anfragen nutzen Hierarchien zu ihrem Vorteil
		\end{itemize}
	\item Stabile Daten
		\begin{itemize}
			\item Daten werden (fast) nie geändert
			\item Nur neue Daten hinzugefügt
		\end{itemize}
	\item Zugriff auf materialisierte Sichten
		\begin{itemize}
			\item Voraggregierte Daten
		\end{itemize}
\end{itemize}
\subsection{Mikro-, Makro- und Meta-Daten}
\begin{itemize}
	\item Mikro-Daten
		\begin{itemize}
			\item Einzelne Observationen, beschreiben Elementarereignisse
			\item Ergebnis der Ladephase $\rightarrow$ Basisdaten
		\end{itemize}
	\item Makro-Daten
		\begin{itemize}
			\item Aggregierte Daten für die Datenanalyse
			\item Ergebnis der Auswertungsphase $\rightarrow$ Data Warehouse, Data Mart
		\end{itemize}
	\item Meta-Daten
		\begin{itemize}
			\item Beschreibungsdaten
			\item Beschreiben die Eigenschaften von Mikro-Daten und Makro-Daten
			\item Beschreiben auch den Entstehungsprozess
		\end{itemize}
\end{itemize}
\subsection{Anforderungen}
\begin{itemize}
	\item Datenwürfel soll flexibel durchsucht werden können
	\item Qualifizierende Daten $\rightarrow$ Dimensionen des Würfels
	\item Quantifizierende Daten $\rightarrow$ Fakten (Zellen des Würfels)
	\item Dimensionen müssen unabhängig sein
	\item Eindeutige Trennung von Fakten
\end{itemize}
\subsection{MD Entwurf}
\begin{enumerate}
  \item Benutzer-Anforderungen
  \item Konzeptionelle Schema
  	\begin{itemize}
		\item semi-formal: mE/R, mUML
	\end{itemize}
  \item Logisches Schema
  	\begin{itemize}
		\item formal: Dimensionen, Cubes
	\end{itemize}
  \item Physisches Schema
  	\begin{itemize}
		\item Relationen, MD-Strukturen
	\end{itemize}
\end{enumerate}
\subsection{Logisches Schema einer Dimension}
\begin{itemize}
	\item Partiell geordnete Menge D von Klassifikationsstufen
	\item Partielle Ordnung erlaubt parallele Hierarchie: \glqq{}Pfade\grqq{}
	\item Orthogonalität: Verschiedene Dimensionen sind unabhänig
\end{itemize}
\subsection{Instanz einer Dimension}
\begin{itemize}
	\item Funktionale Abhängigkeiten $\rightarrow$ Baumstruktur auf Instanzebene
	\item Jeder Pfad im Schema einer Dimension definiert eine Klassenhierarchie
	\item Klassenhierarchie ist ein balancierter Baum
	\item Instanz einer Dimension ist die Menge aller Klassenhierarchien
\end{itemize}
\subsection{Schema eines Datenwürfels}
\begin{itemize}
	\item Definition: Schema eines Datenwürfels $C$
		\begin{itemize}
			\item Struktur: $C[G,M]$
			\item Menge von Fakten: $M = (M_1,\dots , M_m)$
			\item Granularität: $G = (G_1,\dots , G_n)$
			\item Jedes $G_i$ ist ein dimensionales Attribut 
		\end{itemize}
	\item Bsp.:
		\begin{itemize}
			\item Sales und Turnover pro Article, Shop und Day
				\begin{itemize}
					\item $C_\text{sales}[(P.\text{Article}, S.\text{Shop}, T.\text{Day}), (\text{Sales}, \text{Turnover})]$
				\end{itemize}
		\end{itemize}
	\item Fakten (Kenngrößen)
		\begin{itemize}
			\item können auch Eigenschaften zugesprochen werden
			\item sind aber keine Datenstruktur an sich, eher analog zu einem Wertebereich
		\end{itemize}
	\item Aggregatstyp
		\begin{itemize}
			\item nicht-triviale Eigenschafen neben Name und Wertebereich $\rightarrow$ definiert, welche Aggregationsoperationen auf einer Kenngröße ausgeführt werden dürfen
				\begin{enumerate}
					  \item beliebig aggregierbar (Sales, Turnover, \dots): FLOW
					  \item nicht temporal summierbar (Stock, Inventory, \dots): STOCK
					  \item nicht summierbar (Preis, Steuer, i.Allg. Faktoren): VPU (Value per Unit)
				\end{enumerate}
			\item MIN, MAX \& AVG können immer durchgeführt werden
		\end{itemize}
\end{itemize}
\subsection{Instanz eines Würfels}
\begin{itemize}
	\item Alle Zellen aus dem Definitionsbereich des Datenwürfels werden als existierend angenommen, egal ob ein Datensatz physisch vorhanden ist
	\item Gegenüberstellung: Im relationalen Datenmodell herrscht eine andere Grundannahme vor ($\rightarrow$ \glqq{}closed world\grqq{}-Prinzip): Nichts wird angenommen, was nicht explizit als Datensatz vorhanden ist ($\rightarrow$ \glqq{}Intension vs. Extension\grqq{})
\end{itemize}
\subsection{Multidimensionale Operatoren}
\begin{itemize}
	\item Slice
	\item Dice
	\item Drill-Down: Abstieg in der Klassifikationshierarchie zu feinerem Granulat
	\item Roll-Up: Aufstieg in der Klassifikationshierarchie hin zu gröberem Granulat
	\item Drill-Across: Verknüpfung mehrerer Datenwürfel mit gemeinsamen Dimensionen
	\item Drill-Through: Wechsel zu den Originaldaten
	\item Pivotisierung: Wechsel der Darstellung in einer Pivottabelle, entspricht Drehen des Würfels
\end{itemize}
\subsection{Aggregation}
\begin{itemize}
	\item Zusammenfassen mehrerer Zellen
	\item Notwendig beim Roll-Up
	\item Bsp.: vom Tag zum Monat, vom Produkt zur Kategorie
	\item Standart
		\begin{itemize}
			\item SUM
			\item AVG
			\item MIN
			\item MAX
			\item COUNT
		\end{itemize}
	\item Ordnungsbasiert
		\begin{itemize}
			\item cumulating
			\item ranking
		\end{itemize}
\end{itemize}
\subsection{MD Schemaentwurf (Kimball)}
\begin{enumerate}
  \item Auswahl eines Geschäftsprozesses \rightarrow Subjektorientierung
  \item Auswahl der Erfassungsgranularität
  \item Auswahl der Dimensionen
  \item Auswahl der Kennziffern 
\end{enumerate}
\subsection{ROLAP}
\begin{itemize}
	\item Idee für die Speicherung multidimensionaler Daten
		\begin{itemize}
			\item Tabelle mit zusammengesetztem Primärschlüssel aus den Dimensionen
			\item (Nur) für jede vorhandene Datenzelle wird ein Tupel abgespeichert
		\end{itemize}
	\item Trennung von Struktur und Inhalt führt zur Aufteilung in
		\begin{itemize}
			\item zentrale \glqq{}Fact Table\grqq{} und
			\item Dimensionstabellen
		\end{itemize}
\end{itemize}
\subsubsection{Star Schema}
\begin{itemize}
	\item Eine Tabelle für jede Dimension
\end{itemize}
\subsubsection{Snowflake Schema}
\begin{itemize}
	\item Normalisierung der Dimensionstabellen
	\item Viele Tabellen je Dimension
\end{itemize}





















