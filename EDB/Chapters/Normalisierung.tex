\section{Normalisierung}
\subsection{Anomalien}
\begin{itemize}
	\item Einfüge-Anomalie (ohne hinzufügen von Info B, geht Info A nicht)
	\item Lösch-Anomalie
	\item Änderungs-Anomaile
\end{itemize}
\subsection{Funktionale Abhänigkeit $X\to Y$}
$Y$ ist funktional abhängig von $X$, wenn es keine Tupel geben darf, in denen für gleiche $X$-Werte verschiedene $Y$-Werte auftreten

Linke Seite der FA wird \glqq{}Determinante\grqq{} genannt
\subsubsection{Volle Funktionale Abhängigkeit}
$Y$ ist voll funktional abhängig von $X$, wenn es keine echte Teilmenge $Z\subset X$ gibt, für die gilt $Z\to Y$

\subsection{Normalformen}
\subsubsection{Erst Normalform (1NF)}
Eine Relation, die nur atomare Attributwerte besitzt (keine Mengen als Attributwert)
\subsubsection{Zweite Normalform (2NF)}
Eine Relation, in 1NF \& deren Nicht-Schlüsselattribute voll funktional von jedem Schlüsselkandidaten abhängen
\subsubsection{Dritte Normalform (3NF)}
Eine Relation, deren Nicht-Schlüsselkandidaten nicht transitiv abhängig von einem Schlüsselkandidaten sind
\subsubsection{Boyce-Codd-Normalform (BCNF)}
Eine Relation, bei welcher jede Determinante einer FA ein Superschlüssel ist
\subsubsection{Vierte Normalform (4NF)}
Eine Relation $R$ ist in 4NF, wenn für jede nicht-triviale mehrwertige Abhängigkeit $X \twoheadrightarrow A\in R$ gilt: $X$ ist Superschlüssel von $R$

Eine mehrwerte Abhängigkeit gilt, wenn die Attributwerte von $C$ nur von $A$ und nicht von $B$ abhängig sind

$A\twoheadrightarrow C$ ist trivial, wenn $C\in A$ oder $B=\emptyset$

\subsection{Denormalisierung}
Normalisierung kostet Zugriffszeit 

\subsubsection{Wann ist eine Denormalisierung angebracht?}
\begin{itemize}
	\item Seltene Änderungen
	\item Viele Joins
\end{itemize}

Bei weiteren Fragen Anhang VL\_06 lesen