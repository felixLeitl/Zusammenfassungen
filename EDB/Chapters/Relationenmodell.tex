\section{Relationenmodell}
\subsection{Bestandteile eines Datenmodells}
\begin{itemize}
	\item einfache Datentypen und Konstruktoren für zusammengesetzte Datentypen
	\item Konsitenzregeln:
		\begin{itemize}
			\item inhärente Konsistenzregeln:
					
					gelten für ein Datenmodell per Konvention
			\item explizite Konsistenzregeln:
					
					werden für eine Anwendung im Zuge der Datendefinition festgelegt 
		\end{itemize}
	\item Bennenungskonvention für die Bezeichnung von Datenbankelementen
\end{itemize}

\subsection{Begriffe}
\begin{itemize}
	\item Relation: Menge von gleichartig aufgebauten Tupeln
	\item Tupel: Zeile einer Tabelle
	\item Kardinalität: Anzahl der Tupel in einer Relation
	\item Attribut: Spalte einer Tabelle
	\item Grad: Anzahl der Attribute
	\item Relationenschema: 
			\begin{itemize}
				\item Beschreibung einer Relation
				\item besteht aus Relationennamen (z.B. \verb|Personen|)
				\item und einer Menge von Attributen (z.B. \verb|{PNr, Vorname, Nachname}|)
				\item Jedes Attribut wird definiert über einen Attributnamen und einen Wertebereich
				\item z.B. \verb|Personen (PRn, Vorname, Nachname)| 
			\end{itemize}
	\item Relationales Datenbankschema: Menge von Relationalendatenbankschemata
	\item Wertebereich: zulässige Attribute
	\item Superschlüssel: definiert ein Tupel eindeutig
	\item Schlüsselkandidat: Minimaler Superschlüssel
	\item Primärschlüssel: Ausgewählter Schlüsselkandidat
	\item Fremdschlüssel: Attribut, dass mit Primärschlüssel einer Tabelle auf ein bestimmtes Tupel verweist
\end{itemize}
\subsection{Erweiterte Atributdefinition}
\begin{itemize}
	\item NOT NULL
	\item UNIQUE
	\item PRIMARY KEY
\end{itemize}
\subsection{Sicherstellung der Referenziellen Integrität}
\subsubsection{Löschen eines referenzierten Primärschlüssels}
\begin{itemize}
	\item RESTRICTED: ablehnen der Operation
	\item CASCADES: Alle referenzierenden Tupel werden auch gelöscht
	\item NULLIFIE: Referenzen werden auf NULL gesetzt
	\item SET DEFAULT
\end{itemize}
\subsubsection{Ändern eines referenzierten Primärschlüssels}
\begin{itemize}
	\item RESTRICTED
	\item CASCADES
\end{itemize}

\subsection{Integritätsbedingungen}
\subsubsection{\glqq{}System-enforced Integrity\grqq{}}
\begin{itemize}
	\item Primärschlüsseleigenschaft
	\item Referenzielle Integrität
\end{itemize}
\subsubsection{Benutzerdefinierte oder \glqq{}globale\grqq{} Integritätsbedingung}
\begin{itemize}
	\item Bedingungen aus der Anwendungsdomäne, die explizit formuliert werden müssen
	\item Kontrolliert durch das DBMS
	\item Operationen, die die Integritätsbedingungen verletzen werden abgelehnt
\end{itemize}
