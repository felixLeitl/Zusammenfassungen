\section{SQL}
\subsection{Grundstruktur}
\begin{minted}{sql}
SELECT Personalnummer, Name
FROM Mitarbeiter
WHERE Name = 'Müller'
[GROUP BY]
[HAVING]
ORDER BY Geburtsdatum DESC;
\end{minted}
\subsection{Neue Spalten}
\begin{minted}{sql}
SELECT MNR, Gehalt * 1.1 AS Gehaltsprognose
FROM Angestellte;

SELECT MNR, Gehalt + Werbeeinnahmen AS Einkünfte
FROM Angestellte;

(	SELECT Name, Vorname, Gehalt
	FROM Angestellte)
UNION
(	SELECT Name, Vorname, NULL AS Gehalt
	FROM Kunde);
\end{minted}
\subsection{Duplikate}
\begin{minted}{sql}
SELECT DISTINCT Wohnort
FROM Angestellte;
\end{minted}
\subsection{IN}
\begin{minted}{sql}
SELECT * 
FROM Angestellte
WHERE AbtNr IN (6, 4, 2);

SELECT Nachname
FROM Angestellte
WHERE AbtNr IN
	(	
		SELECT AbtNr
		FROM Abteilungen
		WEHRE Ort = 'Erlangen'
	);
\end{minted}
\subsection{EXISTS}
\begin{minted}{sql}
SELECT * 
FROM Angestellte
WHERE EXISTS (SELECT * FROM Abteilungen WHERE Ort = 'Erlangen'); 
\end{minted}
Inner Anfrage muss Bezug zur äußeren Anfrage haben
\subsection{Mengenvergleiche und Quantoren}
\begin{minted}{sql}
SELECT * 
FROM Angestellte
WHERE Wohnort = ANY (SELECT Ort FROM Abteilungen);

SELECT * 
FROM Angestellte
WHERE Gehalt >= ALL (SELECT Gehalt FROM Angestellte);
\end{minted}
\subsection{Join}
\subsubsection{FROM-Klausel}
\begin{minted}{sql}
SELECT PersNr, Wohnort, Bezeichnung
FROM Angestellte, Abteilung
WHERE Angestellte.AbtNr = Abteilung.AbtNr 
	AND Gehalt > 30000
	AND Ort = 'Nürnberg';
\end{minted}
\subsubsection{Auto-Join und Alias-Namen}
\begin{minted}{sql}
SELECT m.Nachname AS Mitarbeiter, v. Nachname AS Chef
FROM Angestellte [AS] m, Angestellte [AS] v
WHERE m.VorgesNr = v.PersNr
AND m.Gehalt > v.gehalt;
\end{minted}
\subsubsection{Cross Join}
\begin{minted}{sql}
SELECT * FROM Angestellte, Abteilung;
SELECT * FROM Angestellte CROSS JOIN ABteilung
\end{minted}
\subsubsection{\Theta-Join}
\begin{minted}{sql}
	SELECT * FROM Angestellte, Abteilungen
	WHERE Angestellte.AbtNr = Abteilungen.AbtNr;
	SELECT * FROM Angestellte JOIN Abteilungen ON Angestellte.AbtNr = Abteilung.AbtNr;
\end{minted}
\subsubsection{Gleichverbund}
\begin{minted}{sql}
SELECT * FROM Angestellte JOIN Abteilung USING (AbtNr);
\end{minted}
\subsubsection{Natürlicher Verbund}
\begin{minted}{sql}
SELECT * FROM Angestellte NATURAL JOIN Abteilungen;
\end{minted}
\subsubsection{Äußerer Verbund}
\begin{minted}{sql}
SELECT * FROM Linke NATURAL LEFT OUTER JOIN Rechte;
SELECT * FROM Linke LEFT JOIN Rechte USING (B);

SELECT * FROM Linke RIGHT JOIN Rechte;

SELECT * FROM Linke NATURAL FULL OUTER JOIN Rechte;
\end{minted}
\subsection{Sortierung}
\begin{minted}{sql}
SELECT *  FROM Angestellte
WHERE AbtNr = 5
ORDER BY Gehalt ASC, Nachname DESC;
\end{minted}
\subsection{Mengenoperationen}
\subsubsection{AVG}
\begin{minted}{sql}
SELECT AVG(Gehalt) FROM Angestellte WHERE ABtNr = 505;
\end{minted}
\subsubsection{COUNT}
\begin{minted}{sql}
SELECT COUNT(DISTINCT Nachname) FROM Angestellte;
SELECT COUNT(*) FROM Angestellte;
\end{minted}
\subsubsection{SUM}
\begin{minted}{sql}
SELECT SUM(Gehalt) FROM Angestellte;
\end{minted}
\subsection{GROUP BY}
%TODO: Understand this
Nötig, wenn man Aggregationen auf Teilmengen durchführt
\begin{minted}{sql}
SELECT AbtNr, AVG(Gehalt)
FROM Angestellte
GROUP BY AbtNr
\end{minted}
\subsection{HAVING}
Einschränkungen nach der Gruppenbildung
\begin{minted}{sql}
SELECT AbtNr, SUM(Gehalt)
FROM Angestellte
GROUP BY AbtNr
HAVING MAX(Gehalt) > 100000 OR MIN(Gehalt) < 20000;
\end{minted}
\subsection{Abarbeitung}
\begin{enumerate}
	\item FROM
	\item WHERE
	\item GROUP BY
	\item HAVING
	\item SELECT
	\item ORDER BY
\end{enumerate}
\subsection{NULL}
\begin{minted}{sql}
SELECT Name
FROM Mitarbeiter
WHERE Gehalt ISNULL
\end{minted}
\subsection{CASE}
\begin{minted}{sql}
SELECT PerNr, Name, 
CASE 
		WHEN Gehalt > 100000 THEN 'Grossverdiener'
		WHEN Quantity > 30000 THEN 'Normalo'
		WHEN Gehalt ISNULL THEN 'unbekannt'
		ELSE 'ARM'
END AS Gehaltsklasse
FROM Angestellte;
\end{minted}
\subsection{WITH}
\begin{minted}{sql}
WITH D AS (SELECT AVG(x.Gehalt) AS Durchschnitt FROM Angestellte x)
SELECT m.Nachname, m.Nachname
FROM Angestellte
WHERE Gehalt > (SELECT Durchschnitt FROM D);
\end{minted}
\subsection{Änderungen in SQL}
\subsubsection{Einfügen}
\begin{minted}{sql}
INSERT INTO Angestellte (PersNr, Nachname, AbtNr)
VALUES (1467, 'Seiler', 505);

CREATE TABLE Spitzenverdiener (
	PersNr		INTEGER not null,
	Nachname	VARCHAR(30),
	Gehalt		INTEGER
);
INSERT INTO Spitzenverdiener
(	SELECT PersNr, Nachname, Gehalt
	FROM Angestellte
	WHERE Gehalt > 100000 );
\end{minted}
\subsubsection{Löschen}
\begin{minted}{sql}
DELETE FROM Angestellte
WHERE PerNr = 704;

DELETE FROM Angestellte
WHERE AbtNr IN (
	SELECT AbtNr
	FROM Abteilungen
	WHERE Ort = 'Trotzdorf'
);
\end{minted}
\subsubsection{Update}
\begin{minted}{sql}
UPDATE Angestellte SET AbtNr = 501
WHERE AbtNr >= 502 AND AbtNr <= 505 AND Gehalt > 60000;

UPDATE Angestellte SET Gehalt = Gehalt * 1.1
WHERE AbtNr IN (
	SELECT AbtNr
	FROM Abteilungen
	WHERE Ort = 'Nürnberg'
)
\end{minted}
\subsection{Integritätsbedingungen}
\begin{minted}{sql}
CREATE TABLE Angestellte (
	PersNr		INT,
	Vorname 	VARCHAR(20),
	Name		VARCHAR(30),
	AbtNr		INT NOT NULL DEFAULT 1,
	VorgesNr	INT,
	CONSTRAINT	MitarbeiterPK
		PRIMARY KEY (PerNr),
	CONSTRAINT VorgesetzterFK
		FOREIGN KEY (VorgesNr) REFERENCES Angestellte (PersNr)
			ON DELETE SET NULL
			ON UPDATE CASCADE,
	CONSTRAINT AbteilungFK
		FOREIGN KEY (AbtNr) REFERENCES Abteilungen (AbtNr)
			ON DELETE SET DEFAULT
			ON UPDATE CASCADE,
);
\end{minted}
\subsection{VIEW}
\begin{minted}{sql}
CREATE VIEW Arme-Angestellte (PerNr, Nachname, Lohn) AS
	SELECT PersNr, Nachname, Gehalt AS Lohn
	FROM Angestellte
	WHERER Gehalt < 400000;
	
SELECT AVG(Lohn)
FROM Arme-Angestellte;
\end{minted}
\subsection{Zugangskontrolle}
\begin{minted}{sql}
GRANT SELECT (PersNr, Nachname, Gehalt), UPDATE (Gehalt)
ON Angestellte TO Schmidt;

REVOKE UPDATE (Gehalt)
ON Angestellte FROM Schmidt
\end{minted}
\subsection{Alter Table}
\begin{minted}{sql}
ALTER TABLE flaeche
ADD COLUMN anteilSportFreizeit REAL CHECK (anteilSportFreizeit >= 0 AND anteilSportFreizeit <= 100);

ALTER TABLE flaeche
ADD COLUMN anteilWald REAL CHECK (anteilWald >= 0 AND anteilWald <= 100);
\end{minted}











