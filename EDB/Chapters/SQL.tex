\section{SQL}
\subsection{Grundstruktur}
%TODO: Finish Grundstruktur
\begin{minted}{sql}
SELECT Personalnummer, Name
FROM Mitarbeiter
WHERE Name = 'Müller'
GROUP BY
HAVING
ORDER BY Geburtsdatum DESC;
\end{minted}
\subsection{Neue Spalten}
\begin{minted}{sql}
SELECT MNR, Gehalt * 1.1 AS Gehaltsprognose
FROM Angestellte;

SELECT MNR, Gehalt + Werbeeinnahmen AS Einkünfte
FROM Angestellte;

(	SELECT Name, Vorname, Gehalt
	FROM Angestellte)
UNION
(	SELECT Name, Vorname, NULL AS Gehalt
	FROM Kunde);
\end{minted}
\subsection{Duplikate}
\begin{minted}{sql}
SELECT DISTINCT Wohnort
FROM Angestellte;
\end{minted}
\subsection{IN}
\begin{minted}{sql}
SELECT * 
FROM Angestellte
WHERE AbtNr IN (6, 4, 2);

SELECT Nachname
FROM Angestellte
WHERE AbtNr IN
	(	
		SELECT AbtNr
		FROM Abteilungen
		WEHRE Ort = 'Erlangen'
	);
\end{minted}
\subsection{EXISTS}
\begin{minted}{sql}
SELECT * 
FROM Angestellte
WHERE EXISTS (SELECT * FROM Abteilungen WHERE Ort = 'Erlangen'); 
\end{minted}
Inner Anfrage muss Bezug zur äußeren Anfrage haben
\subsection{Mengenvergleiche und Quantoren}
\begin{minted}{sql}
SELECT * 
FROM Angestellte
WHERE Wohnort = ANY (SELECT Ort FROM Abteilungen);

SELECT * 
FROM Angestellte
WHERE Gehalt >= ALL (SELECT Gehalt FROM Angestellte);
\end{minted}
\subsection{Join}
\subsubsection{FROM-Klausel}
\begin{minted}{sql}
SELECT PersNr, Wohnort, Bezeichnung
FROM Angestellte, Abteilung
WHERE Angestellte.AbtNr = Abteilung.AbtNr 
	AND Gehalt > 30000
	AND Ort = 'Nürnberg';
\end{minted}
\subsubsection{Auto-Join und Alias-Namen}
\begin{minted}{sql}
SELECT m.Nachname AS Mitarbeiter, v. Nachname AS Chef
FROM Angestellte [AS] m, Angestellte [AS] v
WHERE m.VorgesNr = v.PersNr
AND m.Gehalt > v.gehalt;
\end{minted}
\subsubsection{Cross Join}
\begin{minted}{sql}
SELECT * FROM Angestellte, Abteilung;
SELECT * FROM Angestellte CROSS JOIN ABteilung
\end{minted}
\subsubsection{\Theta-Join}
\begin{minted}{sql}
	SELECT * FROM Angestellte, Abteilungen
	WHERE Angestellte.AbtNr = Abteilungen.AbtNr;
	SELECT * FROM Angestellte JOIN Abteilungen ON Angestellte.AbtNr = Abteilung.AbtNr;
\end{minted}
\subsubsection{Gleichverbund}
\begin{minted}{sql}
SELECT * FROM Angestellte JOIN Abteilung USING (AbtNr);
\end{minted}
\subsubsection{Natürlicher Verbund}
\begin{minted}{sql}
SELECT * FROM Angestellte NATURAL JOIN Abteilungen;
\end{minted}
\subsubsection{Äußerer Verbund}
\begin{minted}{sql}
SELECT * FROM Linke NATURAL LEFT OUTER JOIN Rechte;
SELECT * FROM Linke LEFT JOIN Rechte USING (B);

SELECT * FROM Linke RIGHT JOIN Rechte;

SELECT * FROM Linke NATURAL FULL OUTER JOIN Rechte;
\end{minted}
\subsection{Sortierung}
\begin{minted}{sql}
SELECT *  FROM Angestellte
WHERE AbtNr = 5
ORDER BY Gehalt ASC, Nachname DESC;
\end{minted}
\subsection{Mengenoperationen}
\subsubsection{AVG}
\begin{minted}{sql}
SELECT AVG(Gehalt) FROM Angestellte WHERE ABtNr = 505;
\end{minted}
\subsubsection{COUNT}
\begin{minted}{sql}
SELECT COUNT(DISTINCT Nachname) FROM Angestellte;
SELECT COUNT(*) FROM Angestellte;
\end{minted}
\subsubsection{SUM}
\begin{minted}{sql}
SELECT SUM(Gehalt) FROM Angestellte;
\end{minted}
\subsection{GROUP BY}
%TODO: Understand this
Nötig, wenn man Aggregationen auf Teilmengen durchführt
\begin{minted}{sql}
SELECT AbtNr, AVG(Gehalt)
FROM Angestellte
GROUP BY AbtNr
\end{minted}
\subsection{HAVING}
Einschränkungen nach der Gruppenbildung
\begin{minted}{sql}
SELECT AbtNr, SUM(Gehalt)
FROM Angestellte
GROUP BY AbtNr
HAVING MAX(Gehalt) > 100000 OR MIN(Gehalt) < 20000;
\end{minted}
\subsection{Abarbeitung}
\begin{enumerate}
	\item FROM
	\item WHERE
	\item GROUP BY
	\item HAVING
	\item SELECT
	\item ORDER BY
\end{enumerate}










