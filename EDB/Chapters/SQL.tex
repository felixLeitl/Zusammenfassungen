\section{SQL}
\subsection{Grundstruktur}
\begin{minted}{sql}
SELECT Personalnummer, Name
FROM Mitarbeiter
WHERE Name = 'Müller';
\end{minted}
\subsection{Neue Spalten}
\begin{minted}{sql}
SELECT MNR, Gehalt * 1.1 AS Gehaltsprognose
FROM Angestellte;

SELECT MNR, Gehalt + Werbeeinnahmen AS Einkünfte
FROM Angestellte;

(	SELECT Name, Vorname, Gehalt
	FROM Angestellte)
UNION
(	SELECT Name, Vorname, NULL AS Gehalt
	FROM Kunde);
\end{minted}
\subsection{Duplikate}
\begin{minted}{sql}
SELECT DISTINCT Wohnort
FROM Angestellte;
\end{minted}
\subsection{IN}
\begin{minted}{sql}
SELECT * 
FROM Angestellte
WHERE AbtNr IN (6, 4, 2);

SELECT Nachname
FROM Angestellte
WHERE AbtNr IN
	(	
		SELECT AbtNr
		FROM Abteilungen
		WEHRE Ort = 'Erlangen'
	);
\end{minted}
\subsection{EXISTS}
\begin{minted}{sql}
SELECT * 
FROM Angestellte
WHERE EXISTS (SELECT * FROM Abteilungen WHERE Ort = 'Erlangen'); 
\end{minted}
Inner Anfrage muss Bezug zur äußeren Anfrage haben
\subsection{Mengenvergleiche und Quantoren}
\begin{minted}{sql}
SELECT * 
FROM Angestellte
WHERE Wohnort = ANY (SELECT Ort FROM Abteilungen);

SELECT * 
FROM Angestellte
WHERE Gehalt >= ALL (SELECT Gehalt FROM Angestellte);
\end{minted}
\subsection{Join}
\subsubsection{FROM-Klausel}
\begin{minted}{sql}
SELECT PersNr, Wohnort, Bezeichnung
FROM Angestellte, Abteilung
WHERE Angestellte.AbtNr = Abteilung.AbtNr 
	AND Gehalt > 30000
	AND Ort = 'Nürnberg';
\end{minted}
\subsubsection{Auto-Join und Alias-Namen}
\begin{minted}{sql}
SELECT m.Nachname AS Mitarbeiter, v. Nachname AS Chef
FROM Angestellte [AS] m, Angestellte [AS] v
WHERE m.VorgesNr = v.PersNr
AND m.Gehalt > v.gehalt;
\end{minted}
% TODO: Weitere Joins








