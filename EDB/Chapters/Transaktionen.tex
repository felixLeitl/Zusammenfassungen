\section{Transaktionen}
\subsection{Erwünschte Zustände auf einer Platte}
\subsubsection{Physische Konsistenz}
\begin{itemize}
	\item Korrektheit der Speicherungskonstrukte
	\item Alle Verweise und Adressen (TIDs) stimmen
	\item Alle Indexe sind vollständig und stimmen mit Primärdaten überein
\end{itemize}
\subsubsection{Logische Konsistenz}
\begin{itemize}
	\item Korrektheit der Dateninhalte
	\item Alle im Datenbankschema formulierten Integritätsbedingungen sind erfüllt
\end{itemize}
\subsubsection{Annahmen}
\begin{itemize}
	\item Alle vollständig ausgeführten DB-Operationen hinterlassen einen physisch konsistenten Zustand
	\item Alle vollständig ausgeführten Anwendungsprogramme hinterlassen einen logisch konsistenten Zustand 
\end{itemize}
\subsubsection{Nach einem Fehler}
\begin{itemize}
	\item Die Daten sind i.Allg. weder physisch noch logisch konsistent 
\end{itemize}
\subsubsection{Systemunterstützung}
\begin{itemize}
	\item zur Wiederherstellung eines logischen und physischen konsistenten Zustands der Daten nach einem Fehlerfall
\end{itemize}
\subsubsection{Der herzustellende konsistente Zustand kann sein}
\begin{itemize}
	\item Vor Beginn der Änderungen eines unvollständig ausgeführten Programms
		\begin{itemize}
			\item Rückgängigmachen der bereits ausgeführten Änderungen
		\end{itemize}
	\item Nach Abschluss aller Änderungen eines Programms
		\begin{itemize}
			\item Komplettieren der unvollständigen Änderungen bzw. Wiederholen verlorengegangener Änderungen 
		\end{itemize}
\end{itemize}
\subsubsection{Voraussetzungen}
\begin{itemize}
	\item Geeignete Sicherungs- und Protokollierungsmaßnahmen im laufenden Betrieb (Log)
	\item Protokolldatei
\end{itemize}


