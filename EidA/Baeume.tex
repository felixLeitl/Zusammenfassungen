\section{Bäume}
\subsection{Definitionen}
\subsubsection{Baum}
	Ein Baum besteht aus Knoten, die mit Kanten miteinander Verbunden sind, wobei jeder Knoten, außer der Wurzel genau einen Vorgänger hat
\subsubsection{Wurzelknoten}
	EinWurzelknoten ist ein Knoten auf den keine Kante zeigt
\subsubsection{Blätter}
	Knoten ohne Nachfolger heißen Blätter
\subsubsection{Tiefe}
	Die Tiefe eines Knotens ist die Anzahl der Schritte, die benötigt werden, um von dem Knoten die Wurzel zu erreichen. Die Tiefe eines Baums ist max Tiefe aller Knoten
\subsubsection{Binärbaum}
	Jeder Knoten hat max zwei Nachfolger
\subsubsection{Laufzeit}
	\begin{tabular} {c | c | c | c}
		& Arrays & Listen & Binärbäume \\
		\hline
		Zugriff & $\BigO(1)$ & $\BigO(n)$ & $\BigO(\log n)$ \\
		Einfügen & $\BigO(n)$ & $\BigO(1)$ & $\BigO(\log n)$
	\end{tabular}
\subsubsection{Vollständiger Binärbaum}
	Hat $2^h-1$ innere Knoten und $2^h$ Blätter
	$$
		h=\lceil\log_2(n+1)\rceil-1
	$$
	Höhe:
	$$
		n=2^{h+1}-1
	$$
\subsubsection{Sucheigenschaft}
	Es sei $X$ ein Knoten im binären Suchbaum. Ist $y$ ein Knoten im linken Teilbaum von $x$, dann gilt $y$.Key $\leq$ $x$.Key. Ist $y$ ein Knoten im rechten Teilbaum von $X$ dann gilt: $y$.Key $\leq$ $x$.Key. Dank dieser Eigenschaft lässt sich der Baum inorder durchlaufen
\subsubsection{Heap}
	Ein heap ist ein Binärbaum, der Elemente mit Schlüsseln enthält und in einem Array eingebettet ist. Die Heapbedingung für Max-Heaps fordert:
	\begin{itemize}
	  \item Der Schlüssel eines Knotens ist mindestens so groß wie die Schlüssel seiner Kinder
	  \item Alle Ebenen (bis auf die letzte) sind vollständig gefüllt
	  \item Blätter befinden sich auf einer (max. 2) Ebenen
	  \item Blätter sind linksbündig
	\end{itemize}
	Einbettung ins Array:
	\begin{itemize}
	  \item Wurzel ist a[0]
	  \item linkes Kind von a[i] in a[2$\cdot$i+1]
	  \item rechtes Kind von a[i] in a[2$\cdot$i+2]
	\end{itemize}
\subsubsection{Bruder-Baume}
	Ein binärer Bau, heißt Bruder-Baum, wenn jeder innere Knoten einen oder zwei Nachfolger hat, jeder unäre Knoten einen binären Bruder hat und alle Blätter die selbe Tiefe haben. Ausschließlich binäre Knoten enthalten Schlüssel
\subsubsection{a-b-Baum}
	Jeder innere Knoten hat mindestens a und höchstens b Nachfolger, alle Blätter haben die gleiche Tiefe und jeder Knoten mit $i$ Nachfolgern enthält genau $i-1$ Schlüssel
\subsubsection{B-Baum}
	Ein Baum heißt B-Baum der Ordnung $m$, wenn die folgenden Eigenschaften erfüllt sind:
	\begin{itemize}
	  \item Jeder Knoten außer der Wurzel und der Blätter enthält mindestens [m/2]-1 Daten. Jeder Knoten enthält höchstens m-1 Daten. Die Daten sind sortiert
	  \item Knoten mit $k$ Daten $x_1,...,x_k$ haben $k+1$ Referenzen auf Teilbäume mit Schlüsseln aus den Mengen $\{-\infty,...,x_1-1\},\{x_1+1,...,x_2-1\},...,\{x_{k-1}+1,...,x_{k}-1\},\{x_k+1,...,\infty\}$
	  \item Die Referenzen, die einen Knoten verlassen, sind entweder alle null-Referenzen oder alle Referenzen auf Knoten
	  \item Alle Blätter haben die gleiche Tiefe 
	\end{itemize}
\subsubsection{AVL-Baum}
	Ein binärer Suchbaum heißt AVL-Baum, wenn für jeden Knoten $v$ gilt, dass sich die Höhe des rechten Teilbaums $h(T_r)$ von $v$ und die Höhe des linken Teilbaums $h(T_l)$ von $v$ um maximal 1 unterscheiden:
	\begin{itemize}
	  \item $bal(v)=h(T_r)-h(T_l)\in\{-1,0,+1\}$
	  \item $bal(v)$ gibt den Balancierungsgrad an
	\end{itemize}

\subsection{Binärbaum Operationen}
\subsubsection{Search}
	Vergleicht Knoten mit gesuchtem, wenn größer, dann rechter Teilbaum, wenn kleiner linker Teilbaum
\subsubsection{Min/Max}
	Gibt linkesten Knoten bei Min und rechtesten Knoten bei Max aus
\subsubsection{Vorgänger/Nachfolger}
	Der Nachfolger eines Knoten $x$ in einem binären Suchbaum ist der kleinste Knoten, der größer ist als $x$. 
\subsubsection{Einfügen}
	Finde die Position im Baum und speichere das Element durch Erweiterung des Baums 
\subsubsection{Löschen}
	Fall 1: $z$ hat keine Kinder: \newline
	Entferne den Knoten und setze den Elternknoten auf $\bot$ \newline \newline
	Fall 2: $z$ hat nur ein Kindknoten: \newline
	Entferne $z$ und setzte den Pointer des Elternknoten auf den des Kindknotens von $z$: \newline \newline
	\begin{center}
		\tikz\graph [tree layout, sibling distance=1cm, nodes={circle, draw}]{q--{z--{, r}}};
	\tikz\graph [tree layout, sibling distance=1cm, nodes={circle, draw}]{q--{r}};
	\end{center}
	Fall 3: 
	\begin{center}
		\tikz\graph [tree layout, sibling distance=1cm, nodes={circle, draw}]{z--{l, r--{y--{,x},}}};
		\tikz\graph [tree layout, sibling distance=1cm, nodes={circle, draw}]{z--{l, r--{x,}}};
		\tikz\graph [tree layout, sibling distance=1cm, nodes={circle, draw}]{y--{l, r--{x,}}};
	\end{center}
	Oder
	\begin{center}
		\tikz\graph [tree layout, sibling distance=1cm, nodes={circle, draw}]{z--{l, r--{,x}}};
		\tikz\graph [tree layout, sibling distance=1cm, nodes={circle, draw}]{r--{l, x}};
	\end{center}
\subsection{Heap Operationen}
\subsubsection{Einfügen}
	Füge das Element als Blatt an der ersten freien Stelle ein, ggf. Heap-Eigenschaft, durch aufsteigen wieder herstellen	
	\begin{center}
		\tikz\graph [tree layout, sibling distance=1cm, nodes={circle, draw}]{30--{25--{13--{3,5},16--{27,}},18--{12,8}}};
		\tikz\graph [tree layout, sibling distance=1cm, nodes={circle, draw}]{30--{27--{13--{3,5},25--{16,}},18--{12,8}}};
	\end{center}
\subsubsection{Löschen}
	Knoten entfernen, der größere der Kinder nimmt den Platz ein, ggf. Heap-Eigenschaft wieder herstellen
	
\subsection{a-b-Baum Operationen}
\subsubsection{Einfügen}
	\begin{itemize}
	  \item Man versucht unäre Knoten in binäre Knoten umzuwandeln 
	  \item Gelingt dies nicht muss man einen neuen Knoten zur Wurzel machen
	\end{itemize}
\subsubsection{Löschen}
	Bei dem Löschen geht man vor wie bei den Binärbäumen und stellt anschließend von den Blättern aufsteigend die Baum-Eigenschaften wieder her
\subsection{B-Baum Operationen}
\subsubsection{Einfügen}
	Fall 1: Knoten, in welchem der Schlüssel eingefügt wird wird nicht zu groß: \newline \newline
	Knoten linear durchgehen und an richtiger Stelle Schlüssel mit Nachfolgereferenz auf Null einfügen \newline\newline
	Fall 2: Overflow: \newline \newline
	Knoten spalten und mittleres Element aufsteigen lassen
	\subsubsection{Löschen}
	Fall 1: Schlüssel ist in Blatt und es entsteht kein Underflow\newline \newline
	Schlüssel und Nullpointer löschen \newline \newline
	Fall 2: Schlüssel ist in Blatt aber es entsteht ein Underflow \newline \newline
	Brudertausch: größtes Element an Vorgängerknoten geben und dessen kleinstes an Bruderknoten vererben \newline \newline
	Fall 3: Underflow bei beiden Brüdern \newline \newline
	Verschmelzen: Vorgängerschlüssel, der zwischen den Pointern liegt wird nach unten gezogen und zusammen mit den Brüdern verschmolzen \newline \newline
	\subsection{AVL-Baum Operationen}
	\subsubsection{Einfügen}
		Einfügen wie bei Binärbaum, dann überprüfen, ob der |Balancierungsgrad|>1 \newline \newline
		Fall (-2, -1): Rotation nach rechts \newline \newline
		\tikz\graph [tree layout, sibling distance=1cm, nodes={circle, draw}]{8--{5--{3--{1,4},7},9}};
		\tikz\graph [tree layout, sibling distance=1cm, nodes={circle, draw}]{5--{3--{1,4},8--{7,9}}};
		\newline \newline
		Fall (-2,1): Rotation nach Links-Rechts\newline \newline
		\tikz\graph [tree layout, sibling distance=1cm, nodes={circle, draw}]{15--{5--{3,9--{8,10}},16}};
		\tikz\graph [tree layout, sibling distance=1cm, nodes={circle, draw}]{9--{5--{3,8},15--{10,16}}};
	\subsubsection{Löschen}
		Äquivalent zum Einfügen
		
		
		
		
		
		
		
	
	