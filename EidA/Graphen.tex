\section{Graphen}
	\subsection{Definitionen}
	\subsubsection{Graph}
		Sei $V=\{v_1,...,v_n\}$ eine endliche Menge und $E\subseteq P_2(V)=\{\{u,v\}|u,v\in V,u\not = v\}$. Dann heißt das geordnete Paar $G=(V,E)$ (endlicher, schlichter, ungerichteter) Graph, wobei $V$ die Knotenmenge und $E$ die Kantenmenge von $G$ ist. Ist $e=\{u,v\}\in E$ so sind $u$ und $v$ benachbart (adjazent). $e=\{u,v\}$ oder auch $u-v$ ist dabei eine Kante 
	\subsubsection{Vollständiger Graph}
		Ein Graph heißt vollständig, wenn jede Ecke mit jeder anderen Ecke durch genau eine Kante verbunden ist. $K_n$ bezeichnet den vollständigen Graphen mit $n$ Ecken
	\subsubsection{Grad}
		Der Grad eines Knoten in $G=(V,E)$ ist die Anzahl an benachbarten Knoten. Der Grad eines Graphen entspricht dem maximalen Grad eines enthaltenen Knotens
	\subsubsection{Regulärer Grad}
		Ein Graph heißt regulär, wenn alle Knoten des Graphen den gleichen Grund haben
	\subsubsection{Teilgraph}
		Seien $G$ und $H$ Graphen. $H$ heißt Teilgraph $A$ von $G$, falls $V(H)\subseteq V(G)$ und $E(H)\subseteq E(G)$ gilt
	\subsubsection{Wege und Kreise}
		Ein Weg (auch Pfad genannt) ist eine Folge von Kanten. Ein einfacher Weg besucht keine Knoten doppelt. Ein Kreis ist ein einfacher Weg, wobei Anfangs- und Endknoten äquivalent sind. Ein einfacher Kreis besucht keine Knoten mehrfach
	\subsubsection{Kürzeste Distanz}
		Sei $G=(V,E)$ ein Graph und seien $u,v\in V$ Knoten. Die kürzeste Distanz $\delta(u,v)$ von Knoten $u$ nach Knoten $v$ ist die minimale Anzahl an Knoten in einem Pfad von $u$ nach $v$. Wenn kein Pfad von $u$ nach $v$ existiert, so ist $\delta(u,v)=\infty$. Ein Pfad $\delta(u,v)$ vom $u$ nach $v$ wird als kürzester Pfad bezeichnet
	\subsubsection{Zusammenhängend}
		Graph $G=(V,E)$ heißt zusammenhängend, falls zwischen je zwei Knoten $u,v\in V$ ein Weg existiert
	\subsubsection{Wald}
		Graph $G=(V,E)$ ist ein Wald, falls $G$ keine einfachen Kreise enthält
	\subsubsection{Baum}
		Graph $G=(V,E)$ ist ein Wald, falls G ein zusammenhängender Wald ist
	\subsubsection{Disjunktionen}
		Zwei Graphen sind disjunkt, wenn es keine Weg von dem einen in den anderen gibt. Zwei Wege sind disjunkt, wenn es keinen Knoten gibt, der in beiden enthalten ist
	\subsubsection{Bipartierter Graph}
		Ein einfacher Graph $G=(V,E)$ heißt bipartit, falls sich seine Knoten in zwei disjunkten Teilmengen $A$ und $B$ aufteilen lassen, sodass zwischen den Knoten innerhalb beider Teilmengen keine Kanten verlaufen
	\subsubsection{Zusammenhängende Digraphen}
		Digraph $G=(V,E)$ heißt stark zusammenhängend, falls für je zwei Knoten $u,v\in V$ gilt: Es gibt einen einfachen Weg von $u$ nach $v$ und von $v$ nach $u$. Falls $G$ ungerichtet zusammenhängend ist, ist G schwach zusammenhängend
	\subsubsection{Netzwerke}
		Ein gerichtetes Netzwerk $N=(V,E,c)$ besteht aus einem gerichtetem Graphen $G=(V,E)$ und einer Kapazitätsfunktion $c:E\to \R^+$ so wie zwei dedizierten Knoten $s,t\in V$, genannt die Quelle $s$ und die Senke $t$
	\subsubsection{Fluss}
		Es sei $G=(V,E)$ ein gerichteter Graph, $c:E\to \R^+$ eine Kapazitätsfunktion, $s\in V$ die Quelle und $t\in V$ die Senke. Ein Fluss $F$ in $G$ ist eine Funktion $f:VxV\to \R$ die folgende Eigenschaften erfüllt:
			\begin{enumerate}
			  \item Kapazitätsbegränzung 
			  	$$
			  		\forall u,v\in V \text{ gilt } 0\leq f(u,v)\leq c(u,v)
			  	$$
			  	Fluss positiv und höchstens so groß wie die Kapazität
			  	\item Flusserhaltung
			  	$$
			  		\forall u\in V-\{s,t\}, \text{ gilt } \displaystyle\sum_{v \in V}f(v,u)=\displaystyle\sum_{v\in V}f(u,v)
			  	$$
			  	Der gesamte Fluss in einem Knoten (ohne $s,t$) muss genauso groß sein wie der gesamte Fluss aus diesem Knoten ''Flow in = Flow out'' 
			\end{enumerate}
			\subsubsection{Größe des Flusses}
			Die Größe des ist definiert als die Summe aller Flüsse von der Quelle in den Knoten $v$ minus die Summe aller Flüsse von dem Knoten $v$ zur Quelle
			\subsubsection{Residual-Kapazität}
			Gegebne sei ein Flussnetzwerk $N=(V,E,c)$ mit Quelle $s$, Senke $t$ und Fluss $f$. Betrachte zwei Knoten $u,v\in V$, für diese definieren wir die Restkapazität (auch Residual-Kapazität): \newline
				$$
					c_f(u,v)=\begin{cases}
						c(u,v)-f(u,v) & \text{ falls } (u,v)\in E \\
						f(u,v) & \text{ falls } (v,u) \in E \\
						0 & \text{ sonst }
					\end{cases}
				$$
				\subsubsection{Residual-Netzwerk}
				Gegeben sei ein Flussnetzwerk $N=(V,E,c)$ und ein Fluss $f$. Das Residual-Netzwerk $N_f=(V,E_f,c_f)$ von $N$, welches durch Fluss $f$ aufgespannt wird, ist definiert als:
				$$
					E_F=\{(u,v)\in VxV:c_f(u,v)>0\}
				$$
				\subsubsection{Flusserweiterung}
				Es sei $f$ ein Fluss in $N$ und $N_f$ das zugehörige Residual-Netzwerk. Mit $f\uparrow f'$ bezeichnen wir die Erweiterung des Flusses $f$ um $f'$ als Funktion $VxV\to \R$, definiert durch:
				$$
					(f\uparrow f')(u,v)\begin{cases}
							f(u,v)+f'(u,v')-f'(v,u), & \text{ falls } (u,v)\in E \\
							0 & \text{ sonst}
						\end{cases}
				$$
				\subsubsection{Erweiternder Pfad}
				Es sei $N=(V,E,c)$ ein Flussnetzwerk mit Quelle $s$, Senke $t$ und $f$ ein Fluss in $N$. Ein erweiternder Pfad ist ein Fluss von $s$ zu $t$ in $N_f$
				\subsubsection{Cut}
					Ein Cut$(S,T)$ eines Flussnetzwerks $N=(V,E,c)$ ist eine Partition von $V$ in $S$ und $T$, sodass $s\in S$ und $t\in T$
				\subsubsection{Minimaler Cut}
				Es sei $N=(V,E,c)$ ein Flussnetzwerk und $(S,T)$ ein Cut. Falls $f$ ein Fluss ist, dann ist der Netzfluss $f(S,T)$ durch den Cut $(S,T)$ definiert als $f(S,T)=\displaystyle\sum_{u\in S}\displaystyle\sum_{v\in T}f(u,v)-displaystyle\sum_{u\in S}\displaystyle\sum_{v\in T}f(v,u)$. Mit $c(S,T)=displaystyle\sum_{u\in S}\displaystyle\sum_{v\in T}c(u,v)$ bezeichnen wir die Kapazität des Cuts $(S,T)$. Ein Minimum-Cut eines Netzwerks ist ein Cut mit minimalerKapazität über alle Cuts im Netzwerk
				\subsection{Algorithmen}
	\subsubsection{Breitensuche}
		Idee: 
		\begin{itemize}
		  \item Breitensuche arbeitet iterativ und läuft Level für Level durch den Graphen
		  \item Level für Level bedeutet intuitiv, dass, gegeben ein Startknoten, sich der Algorithmus erst alle Nachbarn des Startknoten anschaut, bevor er die Nachbarn der Nachbarn anschaut, usw.
		\end{itemize}
		Laufzeit: $\BigO(|V|+|E|)$
	\subsubsection{Tiefensuche}
		Idee: Tiefensuche arbeitet im Gegensatz zur Breitensuche nicht Level für Level, sondern steigt immer so weit es geht in die Tiefe
	\subsubsection{Dijkstra}
		Idee:
		\begin{itemize}
		  \item Generalisiert BFS für gewichtete Graphen
		  \item Relaxierugn: Die Pfadlänge zu einem Knoten wird geschätzt und während des Durchlaufs aktualisiert 
		  \item Initial werden die Kosten der nicht zu erreichenden Knoten auf sehr hohe Werte gesetzt (für den Startknoten auf 0) 
		  \item Die Relaxoerung wird so lange. durchgeführt, bis keine Aktualisierung mehr möglich ist
		\end{itemize}
		Laufzeit: $\BigO(|V|^2+|E|)$
	\subsubsection{Floyd-Warshall}
		Idee:
		\begin{itemize}
		  \item Dynamische Programierung
		  \item Wähle einen Knoten $v$ aus und betrachte ob es über $v$ einen kürzeren Weg zwischen allen andern Knotenpaaren gibt: Aktualisiere gegebenenfalls 
		  \item Wiederhole dies für alle Knoten
		\end{itemize}
		Laufzeit: $\Theta(|V|^3)$
	\subsubsection{Page Rank}
	Idee: 
	\begin{itemize}
	  \item Initial ist $r=[\frac{1}{n},...,\frac{1}{n}]^T$
	  \item Wir multiplizieren $r$ nun solange mit $M^T$ bis wir das Ergebnis gut genug approximiert haben
	\end{itemize}
	$$
		r_j=\displaystyle\sum_{i\to j}\beta\frac{r_i}{d_i}+(1-\beta)\frac{1}{n}+(\displaystyle\sum_{i\in D}\beta\frac{r_i}{n})
	$$
	Wobei $r$ der Pagrank, $d$ der Ausgangsgrad, $\beta$ die Sprungwahrscheinlichkeit, $n$ die Anzahl der Knoten und $D$ die Menge der dead ends ist
	\subsubsection{Kruskal}
	Idee: Wähle $n-1$ Kanten mit minimalem Gewicht aus, sodass kein Kreis entsteht:
	\begin{itemize}
	  \item Beginne mit leerer Kantenmenge
	  \item Füge die günstigste Kante hinzu, die keinen Kreis verursacht
	\end{itemize}
	\subsubsection{Prim}
	Idee: Wähle $n-1$ Kanten mit minimalem Gewicht aus, sodass alle Knoten zusammenhängend sind:
	\begin{itemize}
	  \item Wähle eine beliebigen Startknoten $s$
	  \item Füge die günstigste Kante hinzu, die einen neuen Knoten zur Zusammenhangskomponente von $s$ hinzu
	\end{itemize}
	\subsubsection{Rückwärts Kruskal}
	Idee: Entferne $m-n+1$ Kanten mit maximalem Gewicht, sodass alle Knoten zusammenhängen bleiben:
	\begin{itemize}
	  \item Zu Beginn alle Kanten im Graph
	  \item Entferne teuerste Kante, sodass alle Knoten zusammenhängend bleiben
	\end{itemize}
	\subsubsection{Ford-Fulkerson}
	Idee: Der maximale Fluss kann gefunden werden, wen der Fluss im Flussnetzwerk immer wieder durch erweiternde Pfade in $N_f$ erweitert wird

