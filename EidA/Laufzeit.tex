	\section{Laufzeit}
	\subsection{\Theta-Notation}Def. $\Theta(g(n))=\{f(n):\exists c_1,c_2,n_0\geq 0, \text{ sodass } 0\leq c_1\cdot g(n)\leq c_2\cdot g(n)\forall n \geq n_0\}$ \newline \newline
		Bedeutung: 	
		\begin{enumerate}
		\item kleine Werte von $n$ sind nicht. wichtig 
		\item $c_1$ und $c_2$ begrenzen $f$ nach oben und unten
	\end{enumerate} 
	\subsection{$\BigO$-Notation}
		Def. $\BigO(g(n))=\{f(n):\exists c,n_0\geq 0, \text{ sodass }0\leq f(n)\leq c\cdot g(n)\forall n\geq n_0\}$	 \newline \newline
		\begin{enumerate}
			\item $f=\Theta(g(n))\Rarr f(n)=\BigO(g(n))$
			\item $\BigO$-Notation gibt keine exakte obere Schranke an
		\end{enumerate}
		\subsection{\Omega-Notation}
			Def. $\Omega(g(n))=\{f(n):\exists c,n_0\geq 0, \text{ sodass } 0\leq c\cdot g(n)\leq f(n)\forall n\geq n_0\}$ \newline \newline
		\textbf{Theorem.} Für zwei beliebige Funktionen $f(n)$ und $g(n)$ gilt: $f(n) = \Theta(g(n))$ genau dann wenn $f(n)=\BigO(g(n))$ und $f(n)=\Omega(g(n))$
		\subsection{Funktionsklassen}
		\begin{center}
			
		
		\begin{tabular} {c c}
			\hline
			g(n) & \textbf{Wachstum} \\
			\hline
			$1$ & konstant \\
			$\log n$ & logarithmisch \\
			$n$ & linear \\
			$n\log n$ & leicht überlinear \\
			$n^2$ & quadratisch \\
			$n^3$ & kubisch \\
			$n^k$ & polynomiell ($k$=Konstante) \\
			$2^n$ & exponentionell
		\end{tabular}
		\end{center} 
		\subsection{Rechenregeln}
		\subsubsection{Produktregel}
		$$f_1=\BigO(g_1),f_2=\BigO(g_2)\Rarr f_1\cdot f_2 =\BigO(g_1\cdot g_2)$$
		\subsubsection{Summenregel}
		$$f_1=\BigO(g_1),f_2=\BigO(g_2)\Rarr f_1+f_2=\BigO(g_1+g_2)$$
		\subsubsection{Multiplikation mit einer Konstante}		
		Die Konstante fällt weg
		\subsection{Funktionen Vergleichen}
		\subsubsection{Transitivität}
		\begin{align*}
			f(n)=\Theta(g(n)) \text{ und } g(n)=\Theta(h(n)) &\Rarr f(n)=\Theta(h(n)) \\
			f(n)=\BigO(g(n)) \text{ und } g(n)=\BigO(h(n)) &\Rarr f(n)=\BigO(h(n)) \\
			f(n)=\Omega(g(n)) \text{ und } g(n)=\Omega(h(n)) &\Rarr f(n)=\Omega(h(n))
		\end{align*}
		\subsubsection{Reflexivität}
		\begin{align*}
			f(n)&=\Theta(f(n))\\
			f(n)&=\BigO(f(n))\\
			f(n)&=\Omega(f(n)
		\end{align*}
		\subsubsection{Symmetrie}

		$$f(n)=\Theta(g(n))\Leftrightarrow g(n)=\Theta(f(n))$$
		\subsubsection{Transponierende Symmetrie}
		
		$$f(n)=\BigO(g(n))\Leftrightarrow g(n)=\Omega(f(n))$$
		\subsection{Rekursion, Rekurrenz und das Mastertheorem}
		\subsubsection{Rekursion}
		Zusätzliche Kosten durch Divide and Conquer: \newline
		\begin{enumerate}
			\item $T(n): $ Kosten für das Lösen der Instanz der Größe $n$
			\item $f(n): $ Kosten für das Divide einer Instanz der Größe $n$
			\item $g(n): $ Kosten für das Mergen von $k$ Teillösungen
		\end{enumerate}
		Daraus ergibt sich: 
		$$
			T(n) = k \cdot T(m) + f(n) + g(n)
		$$
		Für $T(m)$ ergibt sich:
		$$
			T(m) = k \cdot T(l) + f(m) + g(m)
		$$
		Einsetzen von $T(m)$ in $T(n)$:
		$$
			T(n) = k \cdot (k \cdot T(l) + f(m) + g(m)) + f(n) + g(n) = k^2 \cdot T(l) + k \cdot f(m)+ k \cdot g(m) + f(n) + g(n)
		$$
		\subsubsection{Mastertheorem}
			Gegebne sei eine Rekurrenz der Form $T(n) = aT(\frac{n}{b}+f(n)$ mit Konstanten $a\geq 1$ und $b\geq 1$ sowie einer beliebigen Funktion $f(n)$. $T(n)$ kann asymptotisch wie folgt beschränkt werden:
			\begin{enumerate}
				\item Falls $f(n)=\BigO(n^{\log_b(a)-\epsilon})$, für eine Konstante $\epsilon$, dann gilt $T(n)=\Theta(n^{log_ba})$
				\item Falls $f(n)=\Theta(n^{\log_b(a)})$, dann gilt $T(n)=\Theta (n^{\log_ba}\log n)$
				\item Falls $f(n)=\Omega(n^{\log_b(a)+\epsilon})$, für eine Konstante $\epsilon$ und falls $af(\frac{n}{b})\leq c\cdot f(n)$ für eine Konstante $c<1$ und ab einem $n_0>0$ für alle $n>n_0$, dann gilt $T(n)=\Theta(f(n)$
			\end{enumerate}
