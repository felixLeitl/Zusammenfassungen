\section{Assemblertheorie}
	\subsection{RiscV-Befehlssatz}
		\begin{center}
			\begin{tabular}{c|ccc|c}
			add[i] & x1 & x2 & x3 \\
			and[i] & x1 & x2 & x3 \\
			or[i] & x1 & x2 & x3 \\
			xor[i] & x1 & x2 & x3 \\
			sll[i] & x1 & x2 & x3 & \text{shift left}\\
			srl[i] & x1 & x2 & x3 & \text{shift right}\\
			mv & x1 & x2 & &\text{move}\\ 
			neg & x1 & x2 & &\text{logical negation}\\
			not & x1 & x2 & &\text{bitwise negation}\\
			sub & x1 & x2 & x3 \\
			mul & x1 & x2 & x3 \\
			div & x1 & x2 & x3 \\
			rem & x1 & x2 & x3 &\text{remainder}\\
			li & x1 & \text{Imm} &\\
			la & x1 & \text{lable} &\\
			lb & x1 & \text{Imm}(x2) & & \text{load byte}\\
			lh & x1 & \text{Imm}(x2) &\\
			lw & x1 & \text{Imm}(x2) &\\
			sb & x1 & \text{Imm}(x2) & &\text{store byte}\\
			sh & x1 & \text{Imm}(x2) &\\
			sw & x1 & \text{Imm}(x2) &\\
			call & \text{lable} & &\\
			ret & & &
			\end{tabular}
		\end{center}
	\subsection{Speicherbereiche}
		\begin{center}
			\begin{tabular}{c|c|c|c|c}
				& & Schreibbar & Ausführbar & Dynamisch wachsend \\
				\hline
				Textsegment & Programmcode & Nein & Ja & Nein \\
				\hline
				Datensegment & Globale Variablen & Ja & Nein & Nein \\
				\hline
				Stack & Lokale Variablen & Ja & Nein & Ja \\
				\hline
				Heap & langlebige Variablen & Ja & Nein & ja
				
			\end{tabular}
		\end{center}
	\subsection{Stack}
		In RiscV wächst der Stack von oben nach unten. RiscV bietet ein spezielles Stackpointer-Register(sp Register), welches immer auf die Adresse des zuletzt hinzugefügten Elements zeigt
		\subsubsection{Lesen vom Stack}
			Letztes Element des Stacks lesen: \verb|lw t0, (sp)| \newline
			Vorletztes Element des Stacks lesen: \verb|lw t0, 4(sp)| (hier ein Integer (word))
		\subsubsection{Schreiben auf den Stack}
			Ein Element auf den Stack legen: \newline
			\verb|li t0, 10|\newline
			\verb|addi sp, sp, -4|\newline
			\verb|sw, t0, (sp)|
			\newline \newline
			Mehrere Elemente auf den Stack legen: \newline
			\verb|addi sp, sp,  -12|\newline
			\verb|sw, t0, 8(sp)|\newline
			\verb|sw, t1, 4(sp)|\newline
			\verb|sw, t2, (sp)|
		\subsection{Speicher freigeben}
			\verb|addi sp, sp, 4|