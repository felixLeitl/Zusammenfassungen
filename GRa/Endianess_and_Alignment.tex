\section{Endianess and Alignment}
	\subsection{Endianess}
		Die Endianess beschreibt, in welcher Reihenfolge die Bytes innerhalb zusammenhängender Datums abgespeichert werden\newline
		\textbf{Little Endian}: Least Significant Byte first \newline
		\textbf{Big Endian}: Most Significant Byte first 
	\subsection{Alignment}
		Um sicherzustellen, dass der Zugriff auf ein Datum möglichst wenig Speicherzugriffe benötigt, ist richtiges Alignment nötig. Ein Datum ist korrekt alignd, wenn gilt: 
		$$
			\text{Adresse(Datum)}\%\text{Größe(Datum)}=0
		$$
		Um Daten korrekt auszurichten muss man unter Umständen Padding einfügen, also freien Platz. \newline \newline
		Für \textbf{structs} gilt zusätzlich:
		$$
			\text{Adresse(struct Anfang)}\%\text{max(Größe(DatumInStruct))}=0
		$$
		und am Ende eines structs muss so vile Padding eingefügt werden, sodass, würde das gleiche struct noch einmal abgelegt werden, es automatisch aligned wäre