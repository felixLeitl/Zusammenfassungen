\section{Rechnerarchitektur}
	\subsection{Endo- vs. Befehlsarchitektur}
		Exteren Sicht(Befehlsarchitektur): Was muss nach außen hin sichtbar sein, damit man den Computer programmieren kann? \newline \newline
		Interne Sicht(Endoarchitektur): Wie werden die Funktionalitäten intern realisiert?
	\subsection{Von-Neumann, URA und ISA}
		7 Eigenschaften des URAs/von-Neumann Architektur:
		\begin{enumerate}
			\item Rechner besteht aus 4 Werken:
				\begin{enumerate}
					\item Rechenwerk
					\item Speicherwerk
					\item Ein-/Ausgabewerk
					\item Leitwerk
				\end{enumerate}
			\item Rechner ist programmgestuerert
			\item Programme und Daten im selben Speicher
			\item Hauptspeicher ist in Zellen gleicher Größe aufgeteilt, jede Zelle hat eine Adresse
			\item Programm ist eine Sequenz an Befehlen
			\item Abweichung von sequentieller Ausführung durch Sprünge möglich
			\item Rechner verwendet Binärdarstellung
		\end{enumerate}
	\subsection{Befehlszyklus}
		von-Neumann-Befehlszyklus:
		\begin{enumerate}
			\item Befehl holen
			\item Befehl dekodieren
			\item Operanden holen
			\item Befehl ausführen
			\item Ergebnis zurückschreiben
			\item Nächsten Befehl addresieren
		\end{enumerate}
