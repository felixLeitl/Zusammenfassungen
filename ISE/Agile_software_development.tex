\section{Agile software development}
\begin{table}[H]
\caption{diff. between plan-driven and agile}
Pland-driven: \newline
\includegraphics[scale=0.125]{plan_driven.png}
Agile: \newline
\includegraphics[scale=0.125]{agile.png}
\end{table}
\subsection{Agile manifesto}
\begin{enumerate}
	\item Individuen und Interaktionen über Prozessen und Werkzeugen
	\item Funktionierende Software über akribischer Dokumentation
	\item Zusammenarbeit mit dem Kunden über Vertragsverhandlungen
	\item Auf Veränderungen eingehen über Plan folgen
\end{enumerate}
\begin{table}[H]
\caption{Generic model}
\includegraphics[scale=0.125]{generic_model.png}
\end{table}
\subsection{Rational Unified Process (RUP)}
\subsubsection{Vier Phasen}
\begin{itemize}
	\item Inception: Beginn des Projekts, Business-Modell, grundsächliche Anforderungen und Bedingungen werden definiert
	\item Elaboration: Anforderungen spezifizieren, Architektur, Design und Iterationen 
	\item Construction: Komponenten werden entwickelt und getestet
	\item Transition: Erschaffung von \glqq{}artifacts\grqq{} und Konfiguration, Veröffentlichung 
\end{itemize}
Jede Phase kann mehrfach Wiederholt werden, wird von einem Meilenstein abgeschlossen, liefert \glqq{}artifacts\grqq{}, welche Ergebnisse früher spezifizierter Aktivitäten sind und wird Wiederholt, wenn die \glqq{}artifacts\grqq{}  nicht ausgeliefert werden oder die Standards nicht erreichen \newline \newline
\subsubsection{Disziplinen}
Engineering Workflow
\begin{itemize}
	\item Business modeling:
		\begin{itemize}
			\item Allgemeines Verständnis aller \glqq{}stakeholders\grqq{} der Software
			\item z.B. Komponenten-Diagramme, Use-Case-Diagramme, Klassen-Diagramme
		\end{itemize}
	\item Requirements: Detaillierte Spezifikation des initialen Use-Case und Buisnessmodelle
	\item Analysis \& Design:
		\begin{itemize}
			\item Architektur des Systems wird aus Anforderungen  erarbeitet
			\item Architektur-, Design- und Testdokumente
			\item Klassen- und Zusammenhangsdiagramme
		\end{itemize}
	\item Implementation: Definiert, wie Komponenten implementiert, getestet und integriert werden
	\item Test:
		\begin{itemize}
			\item Beginnt früh im Projekt
			\item Erhöht Verständnis des Systems
			\item Wird ausgeführt, sobald Komponenten, Subsysteme und System verfügbar ist 
		\end{itemize}
	\item Deployment: Finalisieren und veröffentlichen des Produkts
\end{itemize}
Supporting Workflow
\begin{itemize}
	\item Configuration \& change management:
		\begin{itemize}
			\item Organisiert Konfigurations- und Versionsmanagement
			\item Versionsmanagement der Dokumentation
		\end{itemize}
	\item Project management: 
		\begin{itemize}
			\item Planung und Koordination des Projekts
			\item Steuert Resourcen, Qualität und Quantität
			\item Entscheidet, ob zusätzliche Wiederholungen notwendig sind
		\end{itemize}
	\item Environment: Definiert verfügbare Resourcen für die Entwicklungsteams 
\end{itemize}
\begin{multicols}{2}
$\bold{Pros}$:
\begin{itemize}
	\item Definiert Produkt, Rollen und Aktivitäten
	\item Für Großprojekte geeignet
	\item Umfangreiche Nutzung der UML, um echte Szenarien zu modellieren
\end{itemize}	
\columnbreak
$\bold{Cons}$:
\begin{itemize}
	\item Komplex
	\item Nicht flexibel
	\item Große Anzahl an Dokumenten
\end{itemize}
\end{multicols}
\subsection{Kanban}
Kanban verwendet eine visuelles, pull-basiertes System, um den Flow zu optimieren
\subsubsection{Practices}
\begin{itemize}
	\item Workflow definieren und visualisieren
	\item Aktiv Pakete im Workflow managen
	\item Workflow verbessern
\end{itemize}
\begin{table}[H]
\caption{Kanban board}
\includegraphics[scale=0.125]{Kanban_board.png}
\end{table}
Damit Kanban funktioniert muss ein einheitliches Verständnis des Workflows existieren ($\bold D$efinition $\bold o$f $\bold W$orkflow)
\begin{itemize}
	\item \glqq{}Work items\grqq{} definieren - individuelle Einheiten, welche durch den Workflow wandern
	\item Definieren, wann \glqq{}work items\grqq{} beginnen und enden
	\item Definieren, durch welche Zustände \glqq{}work items\grqq{} gehen
	\item Definieren, wie \glqq{}$\bold W$ork $\bold I$n $\bold P$rogress\grqq{} kontrolliert wird
	\item Explizite Richtlinien für die \glqq{}Wanderung\grqq{} durch die Zustände
	\item Ein \glqq{}$\bold S$ervice $\bold L$evel $\bold E$xpectation\grqq{}: Erwartete Durchlaufzeit, jedes Itmes
\end{itemize}
Der Mindestsatz der Durchlaufmetriken besteht aus:
\begin{itemize}
	\item WIP: Anzahl der Items, die begonnen, aber nicht abgeschlossen wurden
	\item Throughput: Anzahl der abgeschlossenen Items pro Zeiteinheit
	\item Work Item Age: Deltazeit seit Bearbeitungsbeginn
	\item Cycle Time: Deltazeit zwischen Bearbeitungsbeginn und Abschluss
\end{itemize}
\subsection{Extreme programming}
\subsubsection{Core values}
\begin{itemize}
	\item Kommunikation
	\item Einfachheit (Simplicity)
	\item Feedback
	\item Mut (Rewrites und ehrliche Kundenkommunikation)
	\item Respekt
\end{itemize}
\subsubsection{Vier Aktivitäten}
\begin{itemize}
	\item Planning
	\item Design
	\item Implementation
	\item Testing
\end{itemize}
\subsubsection{Planning}
\begin{itemize}
	\item $\bold{User}$ $\bold{Stories:}$
		\begin{itemize}
			\item Kurz beschreiben, was passieren soll
			\item Wird für Risikoanalysen, Aufwandsschätzung und Wiederholungsplanung verwendet
			\item Von Firmenseite entwickelt, mit Hilfe der Entwickler
		\end{itemize}
	\item $\bold{Planning}$ $\bold{Game:}$ 
		\begin{itemize}
			\item Endet in einem Veröffentlichungsplan
			\item 4 Variablen: Umfang, Resourcen, Zeit und Qualität
			\item User stories werden beauftragt, um ein Projektplan zu erstellen, der jeden zufrieden stimmt
			\item Ein Veröffentlichungsplan besteht aus 80 $\pm$ 20 user stories
		\end{itemize}
	\item $\bold{Iteration}$ $\bold{Planning:}$
		\begin{itemize}
			\item Iterationen werden mit dem Kunden geplant
			\item User stories werden ausgewählt
			\item Implementierungsaufgaben und Testfälle werden den User stories entnommen
		\end{itemize}
	\item $\bold{Small}$ $\bold{releases}$: Schnelleres Feedback der Kunden, jedoch keine unnötigen Veröffentlichungen
	\item $\bold{Stand-Up}$ $\bold{Meeting:}$ Kurzes Meeting, um Status, Probleme und Lösungen zu besprechen
	\item $\bold{Measuring}$ $\bold{project}$ $\bold{progress}$
	\item $\bold{Move}$ $\bold{People}$ $\bold{Around}$
	\item $\bold{Fix}$ $\bold{XP}$ $\bold{when}$ $\bold{it}$ $\bold{breaks}$
\end{itemize}
\subsubsection{Design}
\begin{itemize}
	\item $\bold{Simplicity}$: Finales Design erst kurz vor Ende und Implementation nur wenn nötig
	\item $\bold{Spike}$ $\bold{solutions}$: Aus Prototypen für Implementation Schlüsse ziehen
	\item $\bold{Refactoring}$: Computer Aided Software Engineering
	\item $\bold{CRC}$ $\bold{cards}$:
		\begin{itemize}
			\item Class, Responsibility and Collaboration
			\item Beschreibe eine Klasse mit maximal 4 Sätzen
			\item Ein Satz Klassen pro Wiederholung
			\item Die Verbindung der Klassen bildet ein UML-Diagramm
		\end{itemize}
		\begin{multicols}{2}	
		Front:
		\begin{tabular}{|c|}
			\hline
			Class Name $\vert$ Superclass \\
			\hline
			Responsibility \\
			\hline
			Collaboration \\
			\hline
		\end{tabular}
		\columnbreak
		Back:
		\begin{tabular}{|c|}
			\hline
			Operations \\
			\hline 
			Attributes \\
			\hline
		\end{tabular}
		\end{multicols}
\end{itemize}
\subsubsection{Implementation}
\begin{itemize}
	\item $\bold{Customer}$ $\bold{availability}$
	\item $\bold{Coding}$ $\bold{guidelines}$
	\item $\bold{Test}$ $\bold{first}$: Schreibe die unit tests am Anfang jeder Wiederholung
	\item $\bold{Pair}$ $\bold{Programming}$: Driver/Pilot und Observer
	\item $\bold{Continuos}$ $\bold{Code}$ $\bold{Integration}$
	\item $\bold{Collective}$ $\bold{Code}$ $\bold{Ownership}$: Jeder kann alles verändern
	\item $\bold{Optimize}$ $\bold{last}$
	\item $\bold{No}$ $\bold{Overtime}$
\end{itemize}
\subsubsection{Testing}
\begin{itemize}
	\item $\bold{Unit}$ $\bold{tests}$: Vor jeder Iteration, ersetzt Dokumentation
	\item $\bold{Acceptance}$ $\bold{tests}$: Das ganze System gegen Spezifikationen testen 
\end{itemize}
\subsubsection{Timeline overview}
\begin{table}[H]
\caption{Timeline overview: XP}
\includegraphics[scale=0.125]{Extreme_programming.png}	
\end{table}
\subsection{Scrum}
\subsubsection{Principles}
\begin{itemize}
	\item Transparency 
	\item Inspection
	\item Adaption: Prozess muss angepasst werden um gute Ergebnisse zu erzeugen 
\end{itemize}
\subsubsection{5 Scrum values}
\begin{itemize}
	\item Commitment
	\item Focus
	\item Openness
	\item Respect
	\item Courage
\end{itemize}
\subsubsection{Timeline overview}
\begin{table}[H]
\caption{Scrum}
\includegraphics[scale=0.125]{Scrum.png}
\end{table}
\subsubsection{Roles}
\begin{itemize}
	\item Product Owner
	\item Stakeholder: Customers, user
	\item Development Team
	\item Scrum Master
\end{itemize}
\subsubsection{Processes}
\begin{itemize}
	\item Sprint: Zeitfenster, um ein Increment zu entwickeln (max. 1 Monat)
	\item Sprint Planning: Anforderungen für nächsten Sprint planen
	\item Daily Scrum: 15 min, um Entwicklung des Tages zu planen
	\item Sprint Review: Increment validieren und Backlog anpassen
	\item Sprint Retrospektiv: Verbesserungen für den nächsten Sprint
\end{itemize}
\subsubsection{Responsibilities}
\begin{itemize}
	\item Product Owner:
		\begin{itemize}
			\item Produktziel formulieren und kommunizieren
			\item Eindeutige Produkt-Backlogs erschaffen
			\item Produkt-Backlog-Einträge ordnen
			\item Sicherstellen, dass der Produkt-Backlog transparent und verständlich ist
		\end{itemize}
	\item Development Team:
		\begin{itemize}
			\item Sprint-Backlog erzeugen
			\item Einhaltung der \glqq{}Done\grqq{}-Definition
			\item Tägliche Anpassungen, um Sprintziel zu erreichen
		\end{itemize}
	\item Scrum Master:
		\begin{itemize}
			\item Techniken bereitstellen
		\end{itemize}
\end{itemize} 
\subsubsection{Product Backlog}
\begin{itemize}
	\item eine geordnete Liste aller Produktanforderungen
	\item Die Einzige Quelle für Anfrageänderungen
	\item Niemals Vollständig
	\item Eine Mischung aus Features, Funktionalität, Veränderungen und Fehlern
\end{itemize}
\subsubsection{Sprint Backlog}
\begin{itemize}
	\item Eine Prognose, der Funktionalitäten des nächsten Icrements
	\item Eine Echtzeitvisualisierung, der Arbeit, des Entwicklungsteams
	\item Detaillierter Fortschritt des Sprints
	\item Wird Ausschließlich vom Entwicklungsteam bearbeitet
\end{itemize}
\subsection{Dynamic Systems Development Method}
\subsubsection{Building Blocks}
\begin{itemize}
	\item Philosophy
	\item Eight Principles
	\item Four P's: Process, People, Products, Practices
	\item Common sense and pramatism
\end{itemize}
\subsubsection{Principles}
\begin{enumerate}
	\item Focus on the business need
	\item Deliver on time
	\item Collaborate
	\item Never compromise quality
	\item Build incremental from firm foundations
	\item Develop iteratively
	\item Communicate constantly and clearly
	\item Demonstrate control
\end{enumerate}
\subsection{Crystal}
Crystal Clear ist nicht verwendbar, bei fail-safe und life-critical Systemen
\subsubsection{Colros}
\begin{table}[H]
\caption{Crystal colors}	
\includegraphics[scale=0.125]{Crystal_colors.png}
\end{table}
\subsubsection{Hardnesses}
\begin{tabular}{|c|c|c|}
	\hline
	C & Loss of comfort & Nutzbarkeit ist reduziert, aber Funktionalität noch gegeben \\
	\hline
	D & Loss of discretionary & Nichtkritische Finanzierungen sind verloren \\
	\hline
	E & Loss of essential money & Kritische Finanzierungen sind verloren \\
	\hline
	L & Loss of Life & Menschen sind gestorben \\
	\hline
\end{tabular}
\subsubsection{Crystal family}
\begin{table}[H]
\caption{Crystal family}
\includegraphics[scale=0.125]{Crystal_family.png}
\end{table}
\subsubsection{Principles}
\begin{itemize}
	\item Regular delivery
	\item Reflective improvement: Vorherige Entwicklungsphase reflektieren
	\item Osmotic and condescended communication
		\begin{itemize}
			\item Osmotic: Kommunikation in kleinen Teams
			\item Condensed: Kommunikation in großen Teams durch Bildung kleiner
		\end{itemize} 
	\item Personal security: Ehrlich ohne Angst, Fehler eingestehen
	\item Choose priorities: Management wählt und kommuniziert
	\item Easy communication with user
	\item Good engineering environment: Autotests, regelmäßige Integration
\end{itemize}
\subsubsection{Required practices (Crystal Clear)}
\begin{itemize}
	\item Osmotic communication, ideally the team works in one room
	\item Increment cycles of less than four months
	\item User is involved in Project
	\item Project assignment is defined with high-level system design
	\item Responsibilities need to be clearly defined 
\end{itemize}
\subsubsection{Roles (Crystal Clear)}
\begin{multicols}{2}
$\bold{Most}$ $\bold{important}$	 $\bold{roles}$:
\begin{itemize}
	\item Customer
	\item Experienced user
	\item Senior architect
	\item Designer / Developer
\end{itemize}
\columnbreak
Less important roles
\begin{itemize}
	\item Coordinator
	\item Field expert
	\item Tester
	\item Author
\end{itemize}
\end{multicols}
\subsection{Agility and large systems}
\begin{table}[H]
\caption{Large scale Problems}
\includegraphics[scale=0.125]{large_scale_agile.png}	
\end{table}
Um agile Modelle bei Großprojekten anzuwenden werden diese meist in plan-driven Ansätze integriert.
\newline 
\autoref{tab:17}















