\section{Agile software development}
\begin{table}[H]
\caption{diff. between plan-driven and agile}
Pland-driven: \newline
\includegraphics[scale=0.125]{plan_driven.png}
Agile: \newline
\includegraphics[scale=0.125]{agile.png}
\end{table}
\subsection{Agile manifesto}
\begin{enumerate}
	\item Individuen und Interaktionen über Prozessen und Werkzeugen
	\item Funktionierende Software über akribischer Dokumentation
	\item Zusammenarbeit mit dem Kunden über Vertragsverhandlungen
	\item Auf Veränderungen eingehen über Plan folgen
\end{enumerate}
\begin{table}[H]
\caption{Generic model}
\includegraphics[scale=0.125]{generic_model.png}
\end{table}
\subsection{Rational Unified Process (RUP)}
\subsubsection{Vier Phasen}
\begin{itemize}
	\item Inception: Beginn des Projekts, Business-Modell, grundsächliche Anforderungen und Bedingungen werden definiert
	\item Elaboration: Anforderungen spezifizieren, Architektur, Design und Iterationen 
	\item Construction: Komponenten werden entwickelt und getestet
	\item Transition: Erschaffung von \glqq{}artifacts\grqq{} und Konfiguration, Veröffentlichung 
\end{itemize}
Jede Phase kann mehrfach Wiederholt werden, wird von einem Meilenstein abgeschlossen, liefert \glqq{}artifacts\grqq{}, welche Ergebnisse früher spezifizierter Aktivitäten sind und wird Wiederholt, wenn die \glqq{}artifacts\grqq{}  nicht ausgeliefert werden oder die Standards nicht erreichen \newline \newline
\subsubsection{Disziplinen}
Engineering Workflow
\begin{itemize}
	\item Business modeling:
		\begin{itemize}
			\item Allgemeines Verständnis aller \glqq{}stakeholders\grqq{} der Software
			\item z.B. Komponenten-Diagramme, Use-Case-Diagramme, Klassen-Diagramme
		\end{itemize}
	\item Requirements: Detaillierte Spezifikation des initialen Use-Case und Buisnessmodelle
	\item Analysis \& Design:
		\begin{itemize}
			\item Architektur des Systems wird aus Anforderungen  erarbeitet
			\item Architektur-, Design- und Testdokumente
			\item Klassen- und Zusammenhangsdiagramme
		\end{itemize}
	\item Implementation: Definiert, wie Komponenten implementiert, getestet und integriert werden
	\item Test:
		\begin{itemize}
			\item Beginnt früh im Projekt
			\item Erhöht Verständnis des Systems
			\item Wird ausgeführt, sobald Komponenten, Subsysteme und System verfügbar ist 
		\end{itemize}
	\item Deployment: Finalisieren und veröffentlichen des Produkts
\end{itemize}
Supporting Workflow
\begin{itemize}
	\item Configuration \& change management:
		\begin{itemize}
			\item Organisiert Konfigurations- und Versionsmanagement
			\item Versionsmanagement der Dokumentation
		\end{itemize}
	\item Project management: 
		\begin{itemize}
			\item Planung und Koordination des Projekts
			\item Steuert Resourcen, Qualität und Quantität
			\item Entscheidet, ob zusätzliche Wiederholungen notwendig sind
		\end{itemize}
	\item Environment: Definiert verfügbare Resourcen für die Entwicklungsteams 
\end{itemize}
\begin{multicols}{2}
$\bold{Pros}$:
\begin{itemize}
	\item Definiert Produkt, Rollen und Aktivitäten
	\item Für Großprojekte geeignet
	\item Umfangreiche Nutzung der UML, um echte Szenarien zu modellieren
\end{itemize}	
\columnbreak
$\bold{Cons}$:
\begin{itemize}
	\item Komplex
	\item Nicht flexibel
	\item Große Anzahl an Dokumenten
\end{itemize}
\end{multicols}
\subsection{Kanban}
Kanban verwendet eine visuelles, pull-basiertes System, um den Flow zu optimieren
\subsubsection{Practices}
\begin{itemize}
	\item Workflow definieren und visualisieren
	\item Aktiv Pakete im Workflow managen
	\item Workflow verbessern
\end{itemize}
\begin{table}[H]
\caption{Kanban board}
\includegraphics[scale=0.125]{Kanban_board.png}
\end{table}








