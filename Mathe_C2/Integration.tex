\section{Integration}
	\subsection{Treppenfunktion}
		\begin{Definition} [ Treppenfunktion]
				Eine Funktion $f[a, b]\to\R$ heißt Treppenfunktion, wenn es eine Unterteilung
				$$
					a=t_0<t_1<...<t_n=b
				$$
				und Konstanten $c_1,...,c_n\in\R$ gibt mit $f(x)=c_k$ falls $x\in(t_{k-1},t_k)$ bzw.
				$$
					f|_{t_{k-1},t_k}=c_k\quad k=1,...,n
				$$
				Im Folgenden sei $\tau[a, b]$ die Menge aller Treppenfunktionen, d.h.
				$$
					\tau[a, b]:=\{f:[a, b]\to\R|f\text{ ist Treppenfunktion}\}
				$$
		\end{Definition}
		\begin{Satz} [ ]
			Die Menge aller Treppenfunktionen $\tau[a, b]$ ist ein Unterraum von $\R^{[a, b]}$
		\end{Satz}
		\begin{Satz} [ ]
			Ein Integral einer Treppenfunktion ist unabhängig von der Unterteilung und damit wohldefiniert
		\end{Satz}
	\subsection{Eigenschaften des Integrals von Treppenfunktionen}
		\begin{Satz} [ ]
			Das Integral $\int_a^b(\cdot)dx:\tau[a, b]\to\R$ ist linear und monoton, d.h.
			\begin{align*}
				\int_a^b(f+g)(x)dx &=\int_a^bf(x)dx+\int_a^bg(x)dx & \forall f, g\in\tau[a, b] \\
				\int_a^b(\lambda f)(x)dx &=\lambda\int_a^bf(x)dx & \forall f\in\tau[a, b],\lambda\in\R \\
				\int_a^bf(x)dx &\leq\int_a^bg(x)dx & \forall f, g\in\tau[a, b] \text{ mit }f\leq g
			\end{align*}
		\end{Satz}
	\subsection{Unter- und Obersumme}
		\begin{Definition} [ Unter- und Obersumme]
			Für beschränktes $f:[a, b]\to\R$ definieren wir die Unter- und Obersumme durch:
			\begin{align*}
				\int_{a*}^bf(x)dx&:=\sup\{\int_a^b\varphi(x)dx|\varphi\in\tau[a, b],\varphi\leq f\},\\
				\int_a^{b*}f(x)dx&:=\inf\{\int_a^b\psi(x)dx|\psi\in\tau[a, b],f \leq\psi\}
			\end{align*}
		\end{Definition}
	\subsection{Riemann-integrierabe Funktionen}
		\begin{Definition} [ Riemann-integrierbar]
			Eine beschränkte Funktion $f:[a, b]\to\R$ heißt Riemann-integrierbar, wenn Unter- und Obersummen übereinstimmen, d.h. wenn
			$$
				\int_{a*}^bf(x)dx=\int_a^{b*}f(x)dx
			$$
			gilt. In diesem Fall setzen wir
			$$
				\int_a^bf(x)dx:=\int_{a*}^bf(x)dx=\int_a^{b*}f(x)dx
			$$
		\end{Definition}
	\subsection{Einschließung zwischen Treppenfunktionen}
		\begin{Satz} [ ]
			Eine Funktion $f:[a, b]\to \R$ ist genau dann Riemann-integrierbar, wenn zu jedem $\epsilon>0$ Treppenfunktionen $\varphi, \psi\in\tau[a, b]$ existieren mit
			$$
				\varphi\leq f\leq\psi \text{ und } \int_a^b\psi(x)dx-\int_a^b\varphi(x)dx\leq\epsilon
			$$
		\end{Satz}
	\subsection{Integriebarkeit stetiger Funktionen}
		\begin{Satz} [ ]
			Jede stetige Funktion $f:[a, b]\to\R$ ist Riemannintegrierbar
		\end{Satz}
		\begin{Satz} [ ]
			Sei $f: [a, b]\to\R$ gleichmäßig stetig. Dann existieren zu jedem $\epsilon>0$ Treppenfunktionen $\varphi,\psi\in\tau[a, b]$ mit
			$$
				\varphi\leq f\leq\psi \text{ und } ||\varphi-\psi||_\infty\leq\epsilon
			$$
		\end{Satz}
		\begin{Satz} [ ]
			Jede stetige Funktion $f:[a, b]\to\R$ auf einem kompakten Intervall $[a, b]$ ist gleichmäßig stetig
		\end{Satz}
		Bemerkung:\newline
		Eine monotone Funktion $f:[a, b]\to\R$ ist automatisch beschränkt
		\begin{Satz} [ ]
			Jede monotone Funktion $f:[a, b]\to\R$ ist Riemann-integrierbar
		\end{Satz}
	\subsection{Eigenschaften des Integrals I}
		\begin{Satz} [ ]
			Das Integral $\displaystyle\int_a^b(\cdot)dx$ ist linear und monoton. D.h. für jede integrierbare Funktionen $f, g:[a, b]\to\R$ und $\lambda\in\R$ gilt:
			\begin{itemize}
				\item $f+g:[a, b]\to\R$ ist integrierbar und es gilt
					$$
						\int_a^b(f+g)(x)dx=\int_a^bf(x)dx+\int_a^bg(x)dx
					$$
				\item $\lambda f:[a, b]\to\R$ ist integrierbar und es gilt
					$$
						\int_a^b(\lambda f)(x)dx=\lambda\int_a^bf(x)dx
					$$
				\item Aus $f\leq g$ folgt
					$$
						\int_a^bf(x)dx\leq\int_a^bg(x)dx
					$$
			\end{itemize}
		\end{Satz}
	\subsection{Eigenschaften des Integrals II}
		\begin{Definition} [ $f_+, f_-,|f|$]
			Für $f:[a, b]\to\R$ definieren wir $f_+, f_-, |f|:[a, b]\to\R$ durch
			$$
				f_+(x)=\begin{cases}
					f(x) & \text{ falls } f(x)>0, \\
					0 & \text{ falls } f(x)\leq0
				\end{cases} \quad
				f_-(x)=\begin{cases}
					-f(x) & \text{ falls } f(x)<0, \\
					0 & \text{ falls } f(x) \geq 0
				\end{cases}
			$$
			sowie $|f|=f_++f_-$
		\end{Definition}
	\subsection{Stückweise Integration} 
		\begin{Satz} [ ]
			Sei $f:[a, b]\to\R$ und $a<c<b$. Dann gilt: $f$ ist genau dann integrierbar, wenn $f|_{[a, c]}$ integrierbar über $[a, c]$ und $f|_[c, b]$ integrierbar über $[c, b]$ ist. \newline
			In diesem Fall gilt
			$$
				\int_a^bf(x)dx=\int_a^cf(x)dx+\int_c^bf(x)dx
			$$
		\end{Satz}
	\subsection{Mittelwertsatz der Integralrechnung}
		\begin{Satz} [ ]
			Sei $f:[a, b]\to\R$ stetig und $\varphi:[a, b]\to\R_0^+$ integrierbar. Dann existiert ein $\xi\in[a, b]$ mit
			$$
				\int_a^bf(x)\varphi(x)dx=f(\xi)\int_a^b\varphi(x)dx
			$$
			Im Spezialfall $\varphi=1$ ergibt sich:
			$$
				\int_a^bf(x)dx=f(\xi)(b-a)
			$$
		\end{Satz}
		\begin{Definition} [ Mittelwert]
			Der Mittelwert einer integrierbaren Funktion $f:[a, b]\to\R$ ist $\frac{1}{b-a}\int_a^bf(x)dx$
		\end{Definition}
		Bemerkung:\newline
		Der Mittelwert gilt nicht, wenn $f$ nur integrierbar, aber nicht stetig ist
	\subsection{Stammfunktionen}
		\begin{Satz} [ ]
			Sei $I$ ein Intervall und $f:I\to\R$ stetig sowie $a\in I$ fest. Definiere $F:I\to\R$ durch
			$$
				F(x):=\int_a^xf(t)dt
			$$
			Dann ist $F:I\to\R$ differenzierbar und es gilt $F'=f$
		\end{Satz}
		\begin{Satz} [ Fundamentalsatz der Differential- und Integralrechnung]
			Sei $f:I\to\R$ stetig und $F$ eine Stammfunktion von $f$. Dann gilt für alle $a, b\in I$
			$$
				\int_a^bf(x)dx=F(b)-F(a)
			$$
		\end{Satz}
	\subsection{Partielle Integration}
		\begin{Satz} [ ]
			Seien, $f, g:[a, b]\to\R$ stetig differenzierbar. Dann gilt:
			$$
				\int_a^bf'(x)g(x)dx=[f(x)g(x)]_a^b-\int_a^bf(x)g'(x)dx
			$$
		\end{Satz}
		Bemerkung: \newline
		\begin{itemize}
			\item Wir schieben eine Ableitung von $f$ auf $g$
			\item Wir müssen eine Stammfunktion von $f'$ kennen
			\item Das Integral auf der rechten Seite sollte \glqq{}einfacher\grqq{} sein
			\item Achtung: Die Änderung des Vorzeichens beachten
		\end{itemize}
	\subsection{Substitutionsregel}
		\begin{Satz} [ ]
			Sei $I$ ein Intervall, $f:I\to\R$ stetig und $\varphi:[a, b]\to I$ stetig differenzierbar. Dann gilt
			$$
				\int_{\varphi(a)}^{\varphi(b)}f(x)dx=\int_a^bf(\varphi(t))\varphi'(t)dt
			$$
		\end{Satz}
		Bemerkung: \newline
		\begin{itemize}
			\item Identifiziere den zu substituierenden Ausdruck
			\item Benenne den Ausdruck $y=\varphi(x)$
			\item Berechne die Ableitung
				$$
					\frac{dy}{dx}=\varphi'(x)
				$$
			\item Verwende $dy=\varphi'(x)dx$
			\item Ersetze $dx=\frac{1}{\varphi'(x)}dy$
			\item Ersetze in allen restlichen Ausdrücken $x$ durch $\varphi^{-1}(y)$
		\end{itemize}
	\subsection{Riemannsche Summen}
		\begin{Definition} [ Unterteilung mit Feinheit]
			Für ein Intervall $[a, b]$ heißt $a=t_0<t_1<...<t_n=b$ eine Unterteilung mit Feinheit $h=\max\{t_k-t_{k-1}|k=1,...,n\}$
		\end{Definition}
		\begin{Definition} [ Riemannsche Summe]
			Sei $f:[a, b]\to\R$ integrierbar und $a=t_0<...<t_n=b$ eine Unterteilung mit Feinheit $h$. Sei ferner in jedem Teilintervall ein Punkt $\xi_k\in[t_{k-1},t_k]$ gegeben. Dann heißt
			$$
				S(f,(t_0,...,t_n),(\xi_0,...,\xi_n))=\sum_{k=1}^nf(\xi_k)(t_k-t_{k-1})
			$$
			eine Riemansche Summe der Feinheit $h$
		\end{Definition}
		\begin{Satz} [ ]
			Sei $f:[a, b]\to\R$ integrierbar. Dann existiert zu jedem $\epsilon>0$ ein $h>0$, sodass für jede Riemannsche Summe $S(f)=S(f,(t_0,...,t_n),(\xi_1,...,\xi_n))$ mit Feinheit $\leq h$ gilt
			$$
				|\int_a^bf(x)dx-S(f)|\leq\epsilon
			$$
		\end{Satz}
		\begin{Satz} [ ]
			Sei $f:[a, b]\to\R$ integrierbar und $(s_n(f))_{n\in\N}$ eine Folge von Riemannschen Summen mit Feinheiten $(h_n)_{n\in\N}$. Dann gilt
			$$
				h_n	\xrightarrow[n\to\infty]{} \quad\Rarr\quad	 S_n(f)\xrightarrow[n\to\infty]{}\int_a^bf(x)dx
			$$
		\end{Satz}
	\subsection{Quadraturformel}
		\begin{Definition} [ Quadraturformel]
			Sei $f:[a, b]\to\R$ integrierbar. Eine Quadraturformel ist eine Summe der Form
			$$
				S_n(f)=\sum_{k=1}^nf(\xi_k)\omega_k
			$$
			mit $\displaystyle\sum_{k=1}^n\omega_k=(b-a)$
		\end{Definition}
		\begin{Satz} [ ]
			 Sei $f:[a, b]\to\R$ Lipschitz-stetig mit Lipschitz-Konstante $L$ (z.B. mit $L=||f'||_\infty$). Dann gilt für jede Riemannsche Summe $S_n(f)$ mit der Feinheit $\leq h$
			 $$
			 	|\int_a^bf(x)dx-S_n(f)|\leq hL(b-a)
			 $$
		\end{Satz}
	\subsection{Quadraturformel mit höherer Ordnung}
		\begin{Satz} [ Fehlerabschätzung]
			Angenommen es gilt:
			\begin{itemize}
				\item Die Zerlegung in Teilintervalle het die Feinheit $\leq h$
				\item $S_N(f)$ integriere in jedem Teilintervall Polynome vom Grad $p$ exakt
				\item $f:[a, b]\to\R$ ist glatt genug
			\end{itemize}
			Dann gilt
			$$
				|\int_a^bf(x)dx-S_n(f)|\leq C(f)h^{p+1}
			$$
		\end{Satz}
	\subsection{Integration und Grenzwertbildung}
		\begin{Satz} [ ]
			Sei $(f_n)$ eine Folge stetiger Funktionen $f_n:[a, b]\to\R$ die gleichmäßig gegen $f:[a, b]\to\R$ konvergiert. Dann gilt
			$$
				\int_a^bf(x)dx=\lim_{n\to\infty}\int_a^bf_n(x)dx
			$$
		\end{Satz}
		Bemerkung: \newline
		Der Satz gilt nicht für punktweise konvergente Funktionsfolgen
		\begin{Satz} [ ]
			Sei $(f_n)$ eine Folge stetig differenzierbarer Funktionen $f_n:[a, b]\to\R$ die punktweise gegen $f:[a, b]\to\R$ konvergiert. Außerdem konvergiere $f'_n:[a, b]\to\R$ gleichmäßig gegen $g:[a, b]\to\R$. Dann ist $f$ differenzierbar und es gilt $f'=g$, d.h.
			$$
				f'(x)=\lim_{n\to\infty} \quad \forall x \in [a, b]
			$$
		\end{Satz}
		\begin{Satz} [ ]
			Sei $f:[a, b]\to\R$ Riemann-integrierbar. Dann gilt
			$$
				\int_a^bf(x)dx=\lim_{z\to b}\int_a^zf(x)dx=\lim_{z\to a}\int_z^af(x)dx
			$$
		\end{Satz}
	\subsection{Integration und gleichmäßige Konvergenz}
		\begin{Satz} [ ]
			Sei $(f_n)$ eine Folge stetiger Funktionen $f_n:[a, b]\to\R$ die gleichmäßig gegen $f:[a, b]\to\R$ konvergiert. Dann gilt
			$$
				\int_a^bf(x)dx=\lim_{n\to\infty}\int_a^bf_n(x)dx
			$$
		\end{Satz}
		Bemerkung: \newline
		Der Satz gilt nicht für punktweise konvergente Funktionsfolgen
	\subsection{Differentation und gleichmäßige Konvergenz}
		\begin{Satz} [ ]
			Sei $(f_n)$ eine Folge stetig differenzierbarer Funktionen $f_n:[a, b]\to\R$ die punktweise gegen $f:[a, b]\to\R$ konvergiert. Außerdem konvergiere $f'_n:[a, b]\to\R$ gleichmäßig gegen $g:[a, b]\to\R$. Dann ist $f$ differenzierbar und es gilt $f'=g$, d.h.
			$$
				f'(x)=\lim_{n\to\infty}f'_n(x)\quad\forall x\in[a, b]
			$$
		\end{Satz}
	\subsection{Uneigentliche Integrale auf unbeschränkten Intervallen}
		\begin{Definition} [ Uneigentliche Integrale erster Art]
			Sei $f:[a, \infty)\to\R$ integrierbar auf jedem Intervall $[a, r]$ mit $a<r<\infty$. Ferner existiere der Grenzwert
			$$
				\lim_{r\to\infty}\int_a^rf(x)dx\in\R
			$$
			Dann heißt das Integral $\displaystyle\int_a^\infty f(x)dx$ konvergent und wir nennen 
			$$
				\int_a^\infty f(x)dx:=\lim_{r\to\infty}\int_a^rf(x)dx
			$$
			das uneigentliche Integral von $f$
		\end{Definition}
		\begin{Definition} [ Uneigentliche Integrale zweiter Art]
			Sei $f:[a, b)\to\R$ integrierbar auf jedem Intervall $[a, r]$ mit $a<r<b$. Ferner existiere der Grenzwert
			$$
				\lim_{r\to b}\int_a^rf(x)dx\in\R
			$$
			Dann heißt das Integral $\displaystyle\int_a^bf(x)dx$ konvergent und wir nenne
			$$
				\int_a^bf(x)dx:=\lim_{r\to b}\int_a^rf(x)dx
			$$
			das uneigentliche Integral von $f$
		\end{Definition}
	\subsection{Uneigentliche Integrale mit zwei kritischen Grenzen}
		\begin{Definition} [ Uneigentliches Integral]
			Sei $a\in\R\cup\{-\infty\},b\in\R\cup\{\infty\}$ und $a<b$. Die Funktion $f:(a, b)\to\R$ sei integrierbar auf jedem Intervall $[r, s]\subset(a, b)$. Für ein $c\in(a, b)$ seinen die uneigentlichen Integrale
			$$
				\int_a^cf(x)dx \text{ und } \int_c^bf(x)dx
			$$
			konvergent. Dann heißt das Integral $\displaystyle\int_a^bf(x)dx$ konvergent und wir nennen
			$$
				\int_a^bf(x)dx:=\int_a^cf(x)dx+\int_c^bf(x)dx
			$$
			das uneigentliche Integral von $f$
		\end{Definition}
	\subsection{Bestimmte Divergenz}
		\begin{Definition} [ bestimmte Divergenz ($+/-$)]
			Eine Folge $(a_n)$ divergiert bestimmt gegen $+\infty$ (bzw. $-\infty$), wenn gilt:
			$$
				\forall C\in\R\exists n_0\in\N\forall n\geq n_0:\quad a_n\geq C \text{ (bzw. } a_n\leq C \text{)}
			$$
			In diesem Fall schreiben wir $a_n\xrightarrow[n\to\infty]{}\infty$ (bzw. $-\infty$)
		\end{Definition}
		\begin{Definition} [ bestimmte Divergenz $\pm$]
			Eine Funktion $f$ divergiert bestimmt gegen $\pm\infty$ für $x\to z\in\R\cup\{\infty,-\infty\}$, wenn für jede Folge $(x_n)$ mit $x_n\xrightarrow[n\to \infty]{}z$ gilt:
			$$
				f(x_n)\xrightarrow[n\to\infty]{} \pm\infty
			$$
			In diesem Fall schreiben wir $f(x)\xrightarrow[x\to z]{}\pm\infty$. (Analog für $x\searrow z$ und $x\nearrow z$)
		\end{Definition}
	\subsection{Nichtlineare Gleichungen}
		\subsubsection{Bisektion}
			Annahme: \newline
			Sei $f:[a, b]\to\R$ stetig und $f(a)<0<f(b)$
			\begin{enumerate}
				\item Setze das Startintervall auf $[x_0, y_0]$
				\item Für $k=0, ...$
					\begin{enumerate}
						\item Berechne $c_k=\frac{x_k+y_k}{2}$
						\item Wenn $f(c_k)=0$: Fertig
						\item Wenn $f(c_k)<0$: Setze $[x_{k+1},y_{k+1}]=[c_k,y_k]$
						\item Wenn $f(c_k)>0$: Setze $[x_{k+1}, y_{k+1}]=[x_k,c_k]$
					\end{enumerate}
			\end{enumerate}
			\begin{Satz} [ ]
				Entweder der Algorithmus terminiert für ein $k$ mit $f(c_k)=0$ oder die Folgen $(x_k), (y_k), (c_k)$ konvergieren gegen eine Nullstelle $x\in[a, b]$. Insbesondere gelte die Fehlerabschätzung
				\begin{align*}
					|x-x_k|\leq|y_k-x_k|&=2^{-k}|b-a|\\
					|x-c_k|\leq|y_k-x_k|&=2^{-k}|b-a|\\
					|x-y_k|\leq|y_k-x_k|&=2^{-k}|b-a|\\
				\end{align*}
			\end{Satz}
		\subsubsection{Fixpunktiteration}
			Sei $f:I\to I$ gegeben. Finde $x\in I$ mit
			$$
				f(x)=x
			$$
			\begin{Satz} [ ]
				Sei $I$ ein abgeschlossenes Intervall und $f:I\to I$ Lipschitz-stetig mit Lipschitz-Konstante $q<1$. Dann gilt:
				\begin{enumerate}
					\item Es existiert genau ein Fixpunkt $x\in I$ mit $f(x)=0$
					\item Für jedes $x_0\in I$ konvergiert die Fixpunktiteration mit
						$$
							x_{k+1}=f(x_k)
						$$
						gegen $x$
					\item Es gelten die Fehlerabschätzungen
						$$
							|x-x_k|\leq\frac{q}{1-q}|x_k-x_{k-1}|\leq\frac{q^k}{1-q}|x_1-x_0|
						$$
				\end{enumerate}

			\end{Satz}

		
		
		
		
		
		