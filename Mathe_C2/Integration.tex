\section{Integration}
	\subsection{Treppenfunktion}
		\begin{Definition} [ Treppenfunktion]
				Eine Funktion $f[a, b]\to\R$ heißt Treppenfunktion, wenn es eine Unterteilung
				$$
					a=t_0<t_1<...<t_n=b
				$$
				und Konstanten $c_1,...,c_n\in\R$ gibt mit $f(x)=c_k$ falls $x\in(t_{k-1},t_k)$ bzw.
				$$
					f|_{t_{k-1},t_k}=c_k\quad k=1,...,n
				$$
				Im Folgenden sei $\tau[a, b]$ die Menge aller Treppenfunktionen, d.h.
				$$
					\tau[a, b]:=\{f:[a, b]\to\R|f\text{ ist Treppenfunktion}\}
				$$
		\end{Definition}
		\begin{Satz} [ ]
			Die Menge aller Treppenfunktionen $\tau[a, b]$ ist ein Unterraum von $\R^{[a, b]}$
		\end{Satz}
		\begin{Satz} [ ]
			Ein Integral einer Treppenfunktion ist unabhängig von der Unterteilung und damit wohldefiniert
		\end{Satz}
	\subsection{Eigenschaften des Integrals von Treppenfunktionen}
		\begin{Satz} [ ]
			Das Integral $\int_a^b(\cdot)dx:\tau[a, b]\to\R$ ist linear und monoton, d.h.
			\begin{align*}
				\int_a^b(f+g)(x)dx &=\int_a^bf(x)dx+\int_a^bg(x)dx & \forall f, g\in\tau[a, b] \\
				\int_a^b(\lambda f)(x)dx &=\lambda\int_a^bf(x)dx & \forall f\in\tau[a, b],\lambda\in\R \\
				\int_a^bf(x)dx &\leq\int_a^bg(x)dx & \forall f, g\in\tau[a, b] \text{ mit }f\leq g
			\end{align*}
		\end{Satz}
	\subsection{Unter- und Obersumme}
		\begin{Definition} [ Unter- und Obersumme]
			Für beschränktes $f:[a, b]\to\R$ definieren wir die Unter- und Obersumme durch:
			\begin{align*}
				\int_{a*}^bf(x)dx&:=\sup\{\int_a^b\varphi(x)dx|\varphi\in\tau[a, b],\varphi\leq f\},\\
				\int_a^{b*}f(x)dx&:=\inf\{\int_a^b\psi(x)dx|\psi\in\tau[a, b],f \leq\psi\}
			\end{align*}
		\end{Definition}
	\subsection{Riemann-integrierabe Funktionen}
		\begin{Definition} [ Riemann-integrierbar]
			Eine beschränkte Funktion $f:[a, b]\to\R$ heißt Riemann-integrierbar, wenn Unter- und Obersummen übereinstimmen, d.h. wenn
			$$
				\int_{a*}^bf(x)dx=\int_a^{b*}f(x)dx
			$$
			gilt. In diesem Fall setzen wir
			$$
				\int_a^bf(x)dx:=\int_{a*}^bf(x)dx=\int_a^{b*}f(x)dx
			$$
		\end{Definition}
	\subsection{Einschließung zwischen Treppenfunktionen}
		\begin{Satz} [ ]
			Eine Funktion $f:[a, b]\to \R$ ist genau dann Riemann-integrierbar, wenn zu jedem $\epsilon>0$ Treppenfunktionen $\varphi, \psi\in\tau[a, b]$ existieren mit
			$$
				\varphi\leq f\leq\psi \text{ und } \int_a^b\psi(x)dx-\int_a^b\varphi(x)dx\leq\epsilon
			$$
		\end{Satz}
	\subsection{Integriebarkeit stetiger Funktionen}
		\begin{Satz} [ ]
			Jede stetige Funktion $f:[a, b]\to\R$ ist Riemannintegrierbar
		\end{Satz}
		\begin{Satz} [ ]
			Sei $f: [a, b]\to\R$ gleichmäßig stetig. Dann existieren zu jedem $\epsilon>0$ Treppenfunktionen $\varphi,\psi\in\tau[a, b]$ mit
			$$
				\varphi\leq f\leq\psi \text{ und } ||\varphi-\psi||_\infty\leq\epsilon
			$$
		\end{Satz}
		\begin{Satz} [ ]
			Jede stetige Funktion $f:[a, b]\to\R$ auf einem kompakten Intervall $[a, b]$ ist gleichmäßig stetig
		\end{Satz}

		
		
		
		
		
		
		