%TODO: Satz von Weierstrass finden
%TODO: Eigenschaften & Bemerkungen
\section{Stetige Funktionen}
	\begin{Definition}[ Stetig]
		Def: \newline
	Sei $I\subset \R$ ein Intervall und $f: I \to \R$ eine Funktion
	\begin{itemize}
		\item $f$ heißt stetig im Punkt $x\in I$, wenn gilt: \newline
			Für jede Folge $(X_n)$ in $I$ mit $x_n\to x$ gilt auch $f(x_n)\to f(x)$
		\item $f$ heißt stetig, wenn $f$ in jedem Punkt $x \in I$ stetig ist
	\end{itemize}
	\end{Definition}
	Anschaulich:
	\begin{itemize}
		\item \glqq{} $f$ stetig in $x$ \grqq{}  bedeutet, dass $f$ in $x$ nicht springt
		\item \glqq{} $f$ stetig \grqq{} bedeutet, dass $f$ nirgendwo springt 
	\end{itemize}
	\subsection{$\Q$ ist dicht in $\R$}
		\begin{Lemma}[ ]
			Zu jeder reellen Zahl $r\in \R$ und jedem $\epsilon > 0$ existiert eine rationale Zahl $q\in \Q$ mit $|r-q|<\epsilon$
		\end{Lemma}
		\begin{Lemma} [ ]
			Zu jeder reellen Zahl $r\in \R$und jedem $\epsilon>0$ existiert eine rationale Zahl $r\in\R\backslash\Q$ mit $|r-q|<\epsilon$
		\end{Lemma}
		\begin{Lemma} [ ]
			Zu jeder reellen Zahl $x\in\R$ existiert eine Folge $(x_n)$ in $\Q$ mit $x_n\to x$\newline
			Zu jeder rationalen Zahl $x\in \Q$ existiert eine Folge $(x_n)$ in $\R\backslash\Q$ mit $x_n\to x$
		\end{Lemma}
	\subsection{Eigenschaften stetiger Funktionen}
		\begin{Satz} [ ]
			Sei $I$ ein Intervall, $x \in I$ und $f, g: I\to \R$ Funktionen, die stetig in $x$ sind. Dann gilt:
			\begin{itemize}
				\item $f+g$ ist stetig in $x$
				\item $f-g$ ist stetig in $x$
				\item $f\cdot g$ ist stetig in $x$
				\item Falls $g(y)\not = 0, \forall y\in I$, so ist $\frac{f}{g}$ stetig in $x$
			\end{itemize}
		\end{Satz}
	\subsection{Komposition stetiger Funktionen}
		\begin{Satz} [ ]
			Seien $I, J$ Intervalle, $f: I\to\R$ und $g: J\to \R$ und $f(I)\subset J$ \newline 
			Ferner sei $f$ stetig in $x\in I$ und $g$ stetig in $y=f(x)$ \newline
			Dann ist $g\circ f: I\to \R$ stetig in $x$
		\end{Satz}
	\subsection{Zwischenwertsatz}
		\begin{Satz} [ Zwischenwertsatz]
			Sei $f:[a, b]\to\R$ stetig auf dem abgeschlossenen Intervall $[a, b]$. Dann nimmt $f$ in $(a, b)$ jeden beliebigen Wert $y$ zwischen $f(a)$ und $f(b)$ an
		\end{Satz}
		\begin{Satz} [ Variante des Zwischenwertsatz]
			Sei $f:[a, b]\to \R$ stetig auf dem abgeschlossenen Intervall $[a, b]$. Dann nimmt $f$ in $[a, b]$ jeden beliebigen Wert
			$$
				y\in [\min_{x\in[a, b]}f(x), \max_{x\in[a, b]}f(x)]
			$$
			an
		\end{Satz}
	\subsection{Satz über Nullstellen}
		\begin{Satz} [ Nullstellen]
			Sei $f:[a, b]\to \R$ stetig auf dem abgeschlossenen Intervall $[a, b]$ und es gelte $f(a)<0<f(b)$ oder $f(a) > 0 > f(b)$. Dann hat $f$ in $(a, b)$ mindestens eine Nullstelle, d.h. es existiert ein $x\in(a, b)$ mit $f(x)=0$
		\end{Satz}
	\subsection{Satz von Minimum und Maximum}
		\begin{Satz} [ Minimum und Maximum]
			Sei $f:[a, b]\to \R$ stetig auf dem abgeschlossenen Intervall $[a, b]$. Dann nimmt $f$ in $[a, b]$ Maximum und Minimum an, d.h. es existieren $x_{\min}
			, x_{\max} \in [a, b]$ mit 
			$$
				f(x_{\min})\leq f(x) \leq f(x_{\max}, \quad \forall x\in [a, b]
			$$
		\end{Satz}
		Insbesondere gilt für $x_{\min}$ und $x_{\max}$
		\begin{align*}
			f(x_{\min}) &= \inf_{x\in[a, b]} f(x) = \min_{x\in [a, b]} f(x) \\
			f(x_{\max}) & = \sup_{x\in [a, b]} f(x) = \max_{x \in [a, b]} f(x)
		\end{align*}
		\begin{Definition} [ Schreibweisen]
			Sei $(x_n)$ eine reelle Folge. Wir schreiben $x_n\to\infty$, wenn gilt
			$$
				\forall C \in \R\exists n_0\in\N\forall n\geq n_0: x_n\geq C
			$$
			Analog schreiben wir $x_n\to -\infty$, wenn gilt
			$$
				\forall C \in\R\exists n_0\in\N\forall n\geq n_0:x_n\leq C
			$$
		\end{Definition}
	\subsection{Metrik in normierten Räumen}
		\begin{Definition} [ Metrik]
			Ist $(V,||\cdot||)$ ein normierter Raum. Dann heißt die Abbildung
			$$
				d: V \times V \to \R, \quad d(x, y):=||x-y||
			$$
			die zur Norm $||\cdot||$ gehörige Metrik
		\end{Definition}
	\subsection{$\epsilon$-Umgebung}
		\begin{Definition} [ $\epsilon$-Umgebung ]
			Sei $(V, ||\cdot||)$ ein normierter Raum. Für einen Punt $x\in V$ und $\epsilon>0$ heißt die Menge
			$$
				B_\epsilon(x):= \{d(x, y) < \epsilon\} = \{y\in V:||x-y||<\epsilon\}
			$$
			eine $\epsilon$-Umgebung von $x$. Man spricht von der offenen Kugel mit Radius $\epsilon$ um $x$
		\end{Definition}
	\subsection{Umgebungen}
		\begin{Definition} [ Umgebung]
			Sei $(V, ||\cdot||)$ ein normierter Raum und $x\in V$ ein Punkt in V. Dann heißt eine Teilmenge $U\subset V$ eine Umgebung von $x$, wenn sie eine $\epsilon$-Umgebung von $x$ enthält, d.h. wenn $\epsilon > 0$ existiert mit $B_\epsilon(x)\subset U$
		\end{Definition}
	\subsection{Innere Punkte}
		\begin{Definition} [ Innerer Punkt]
			Sei $M\subset V$. Ein Punkt $x\in M$ heißt innerer Punkt von $M$, falls ein $\epsilon>0$ mit $B_\epsilon(x)\subset M$ existiert. \newline
			Die Menge aller inneren Punkte von $M$ heißt das Innere von $M$ und wird mit $\mathring{M}$ bezeichnet
		\end{Definition}
	\subsection{Randpunkte}
		\begin{Definition} [ Randpunkt]
			Sei $M\subset V$. Ein Punkt $x\in V$ heißt Randpunkt von $M$, falls in jeder Umgebung $B_\epsilon(x)$ ein Punkt aus $M$ und aus $V\backslash M$ ist. \newline
			Die Menge aller Randpunkte von $M$ heißt der Rand von $M$ und wird mit $\partial M$ bezeichnet. \newline
			Die Menge $\overline{M}:=M\cup\partial M$ heißt der Abschluss von $M$
		\end{Definition}
	\subsection{Offene und abgeschlossene Mengen}
		\begin{Definition} [ Offene Menge]
			Eine Teilmenge $O\subset V$ heißt offen, wenn zu jedem $x\in O$ ein $\epsilon>0$ mit $B_\epsilon(x)\subset O$ existiert, d.h., wenn $O$ Umgebung aller ihrer Punkte $x\in O$ ist.
		\end{Definition}
		\begin{Definition} [ Abgeschlossene Menge]
			Eine Teilmenge $A\subset V$ heißt abgeschlossen, wenn $V\backslash A$ offen ist
		\end{Definition}
	\subsection{Konvergenz in $\R$}
		\begin{Definition} [ Konvergenz]
			Eine reelle Folge $(x_n)$ konvergiert gegen $x\in\R$, wenn gilt:
			$$
				\forall \epsilon > 0 \exists n_0 \in \N \forall n \geq n_0: \quad |x_n - x| < \epsilon
			$$
			Mit Hilfe der Metrik $d(x, y) = |x-y|$ können wir dies auch formulieren als
			$$
				\forall \epsilon > 0 \exists n_0 \in \N \forall n \geq n_0: \quad d(x_n, x) < \epsilon
			$$
			und mit $\epsilon$-Umgebung als
			$$
				\forall \epsilon > 0 \exists n_0 \in \N \forall n \geq n_0: \quad x_n \in B_\epsilon(x)
			$$
		\end{Definition}
	\subsection{Konvergenzkriterien}
		\begin{Lemma} [ ]
			Sei $(V, ||\cdot||)$ ein normierter Raum $(x_n)$ eine Folge in $V$ und $x \in V$. \newline
			Dann sind äquivalent:
			\begin{enumerate}
				\item $(x_n)$ konvergiert gegen $x$, d.h. $x_n\to x$
				\item $||x_n - x||$ ist Nullfolge, d.h. $||x_n - x||\to 0$
				\item Es gilt $||x_n - x||\geq y_n$ für eine reelle Nullfolge $(y_n)$
				\item Für jede Umgebung $U$ von $x$:
					$$
						\exists n_0 \in \N \forall n \geq n_0: \quad x_n \in U
					$$
			\end{enumerate}
		\end{Lemma}
	\subsection{Äquivalente Normen}
		\begin{Definition} [ Äquivalente Normen]
			Sei $V$ ein $\mathbb{K}$-Vektorraum und $||\cdot||_\alpha$ und $||\cdot||_\beta$ zwei Normen auf $V$. Dann heißen $||\cdot||_\alpha$ und $||\cdot||_\beta$ äquivalent, wenn Konstanten $\alpha, \beta > 0$ existieren mit 
			$$
				\alpha||x||_\alpha \leq ||x||_\beta \leq \beta||x||_\alpha \quad \forall x \in V
			$$
		\end{Definition}
		\begin{Satz} [ ]
			$||\cdot||_1, ||\cdot||_2, ||\cdot||_\infty$ sind äquivalent auf $\R^n$
		\end{Satz}
		\begin{Satz} [ ]
			Sei $V$ ein endlichdimensionaler Vektorraum. Dann sind alle Normen auf $V$ äquivalent
		\end{Satz}
	\subsection{Äquivalente Normen und ihre Umgebungen}
		\begin{Satz} [ ]
			Sei $V, ||\cdot||_\alpha$ ein normierter Raum und $U \subset V$ eine Umgebung von $x$ bezüglich $||\cdot||_\alpha$. Dann ist $U$ auch Umgebung bezüglich jeder zu $||\cdot||_\alpha$ äquivalenten Norm $||\cdot||_\beta$
		\end{Satz}
	\subsection{Konvergenz und äquivalente Normen}
		\begin{Satz} [ ]
			Sei $V$ ein $\mathbb{K}$-Vektorraum, $||\cdot||_\alpha $ und $||\cdot||_beta$ zwei äquivalente Normen. Dann sind für eine Folge $(x_n)$ in $V$ und $x \in V$ äquivalent:
			\begin{itemize}
				\item $(x_n)$ konvergiert gegen $x$ bezüglich $||\cdot||_\alpha$
				\item $(x_n)$ konvergiert gegen $x$ bezüglich $||\cdot||_\beta$
			\end{itemize}
		\end{Satz}
	\subsection{Konvergenz in $\R^n$}
		\begin{Satz} [ ]
			Sei $||\cdot||$ eine Norm auf $\R^n$, $(x^{(n)})_{n\in\N}$ eine Folge in $\R^m$ und $x\in\R^m$. Dann konvergiert $(x^{}(n))$ genau dann gegen $x$, wenn gilt
			$$
				x^{(n)}_k \xrightarrow[n\to\infin]{}x_k \quad k=1,...,m
			$$
		\end{Satz}
	\subsection{Abgeschlossene Mengen und Konvergenz}
		\begin{Satz} [ ]
			Sei $A\subset V$ eine Teilmenge eines normierten Raums, dann sind äquivalent:
			\begin{enumerate}
				\item $A$ ist abgeschlossen
				\item Für jede konvergente Folge $(x_n)$ mit $x_n \in A$ für alle $n$ gilt auch $\lim_{n\to\infty}x_n \in A$
			\end{enumerate}
		\end{Satz}
	\subsection{Grenzwertsätze in normierten Räumen}
		\begin{Satz} [ ]
			Der Grenzwert einer in $V$ konvergenten Folge ist eindeutig bestimmt
		\end{Satz}
		\begin{Satz} [ ]
			Konvergente Folgen sind beschränkt
		\end{Satz}
		\begin{Satz} [ ]
			Sei $V$ ein normierter Raum, $(a_n)$ und $(b_n)$ Folgen in $V$ und $(\lambda_n)$ eine Folge in $\mathbb{K}$ mit 
			$$
				a_n\to a\in V, \quad b_n\to b \in V, \quad \lambda_n \to \lambda \in \mathbb{K}
			$$ 
			Dann gilt:
			\begin{itemize}
				\item $a_n + b_n \to a + b$
				\item $a_n - b_n \to a - b$
				\item $\lambda_n a_n \to \lambda a$
			\end{itemize}
		\end{Satz}
	\subsection{Cauchey-Folgen}
		\begin{Definition} [ Cauchey-Folge]
			Eine Folge $(a_n)$ in $V$ heißt Cauchey-Folge, wenn gilt:
			$$
				\forall \epsilon \exists n_0 \in \N \forall n, m \geq n_0: \quad ||a_n - a_m|| < \epsilon
			$$
		\end{Definition}
		\begin{Satz} [ ]
			Jede Cauchey-Folge in $V$ ist beschränkt
		\end{Satz}
		\begin{Satz} [ ]
			Jede konvergente Folge in $V$ ist eine Cauchey-Folge
		\end{Satz}
	\subsection{Konvergenz von Chauchey-Folgen}
		\begin{Definition} [Vollständig]
			Ein normierter Raum heißt vollständig, wenn jede Chauchey-Folge in $V$ konvergiert
		\end{Definition}
		\begin{Satz} [ ]
			$\R$ ist vollständig
		\end{Satz}
		\begin{Satz} [ ]
			Sei $V$ endlichdimensional. Dann ist $V$ vollständig
		\end{Satz}
	\subsection{Konvergenz und Teilfolgen}
		\begin{Satz} [ ]
			Eine Folge $(a_n)_{n\in \N}$ in $V$ konvergiert genau dann gegen $a$, wenn jede Teilfolge $(a_{n_k})_{k\in\N}$ gegen $a$ konvergiert
		\end{Satz}
		\begin{Satz} [ Bolzano-Weierstrass]
			Sei $V$ endlichdimensional. Dann besitzt jede beschränkte Folge in $V$ eine konvergente Teilfolge
		\end{Satz}
	\subsection{Stetigkeit in normierten Räumen}
		\begin{Satz} [ ]
					Sind $f, g: D\to Y$ sowie $h: D\to \R$ für $D\subset Y$ stetig, dann sind auch $f + g: D\to Y, f-g: D\to Y$ und $hf: D\to Y$ stetig
		\end{Satz}
	\subsection{Stetigkeit auf Unterräumen}
		\begin{Satz} [ ]
			Sei $f: X\to Y$ stetig und $D \subset X$ eine Teilmenge von $X$. Dann sind auch die Einschränkungen $f|_D:D\to Y$ stetig
		\end{Satz}
	\subsection{$\epsilon$-$\delta$-Kriterium}
		\begin{Satz} [ ]
			Eine Funktion $f: D\to Y$ ist genau dann steig im Punkt $x/in D$, wenn gilt:
			$$
				\forall \epsilon > 0\exists\delta>0: \quad ||x-y||_X < \delta \Rarr ||f(x)-f(y)||_Y<\epsilon \quad \forall y\in D
			$$
		\end{Satz}
	\subsection{Gleichmäßig stetig}
		\begin{Definition} [ Gleichmäßigkeit]
			Eine Funktion $f: D\to Y$ heißt gleichmäßig stetig, wenn gilt:
			$$
				\forall \epsilon > 0\exists\delta>0\forall x, y \in D: \quad ||x-y||_X < \delta \Rarr ||f(x)-f(y)||_Y<\epsilon \quad \forall y\in D
			$$
		\end{Definition}
	\subsection{Lipschitz-Stetigkeit}
		\begin{Definition} [ Lipschitz-stetig]
			Eine Abbildung $f: D\to Y$ auf $D\subset X$ heißt Lipschitz-stetig, wenn ein $L\geq 0$ existiert mit 
			$$
				||f(x)-f(y)||_Y\leq L||x-y||_X \quad \forall x, y \in X
			$$
		\end{Definition}
		\begin{Satz}[ ]
			Jede Lipschitz-stetige Abbildung ist gleichmäßig stetig
		\end{Satz}
	\subsection{Stetigkeit linearer Abbildungen}
		\begin{Satz} [ ]
			Sei $A\in \R^{m\times n}$ und $f: \R^n\to \R^m$ mit $f(x)=Ax$. Dann ist $f$ Lipschitz-stetig und somit insbesondere gleichmäßig stetig und stetig
		\end{Satz}
		