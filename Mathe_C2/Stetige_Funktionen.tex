% Satz von Weierstrass finden
\section{Stetige Funktionen}
	\begin{definition}
		Def: \newline
	Sei $I\subset \R$ ein Intervall und $f: I \to \R$ eine Funktion
	\begin{itemize}
		\item $f$ heißt stetig im Punkt $x\in I$, wenn gilt: \newline
			Für jede Folge $(X_n)$ in $I$ mit $x_n\to x$ gilt auch $f(x_n)\to f(x)$
		\item $f$ heißt stetig, wenn $f$ in jedem Punkt $x \in I$ stetig ist
	\end{itemize}
	\end{definition}
	Anschaulich:
	\begin{itemize}
		\item \glqq{} $f$ stetig in $x$ \grqq{}  bedeutet, dass $f$ in $x$ nicht springt
		\item \glqq{} $f$ stetig \grqq{} bedeutet, dass $f$ nirgendwo springt 
	\end{itemize}
	\subsection{$\Q$ ist dicht in $\R$}
		\begin{Lemma}
			Zu jeder reellen Zahl $r\in \R$ und jedem $\epsilon > 0$ exestiert eine rationale Zahl $q\in \Q$ mit $|r-q|<\epsilon$
		\end{Lemma}
		\begin{Lemma}
			Zu jeder reellen Zahl $r\in \R$und jedem $\epsilon>0$ existiert eine rationale Zahl $r\in\R\backslash\Q$ mit $|r-q|<\epsilon$
		\end{Lemma}
		\begin{Lemma}
			Zu jeder reellen Zahl $x\in\R$ existiert eine Folge $(x_n)$ in $\Q$ mit $x_n\to x$\newline
			Zu jeder rationalen Zahl $x\in \Q$ existiert eine Folge $(x_n)$ in $\R\backslash\Q$ mit $x_n\to x$
		\end{Lemma}
	\subsection{Eigenschaften stetiger Funktionen}
		\begin{Satz}
			Sei $I$ ein Intervall, $x \in I$ und $f, g: I\to \R$ Funktionen, die stetig in $x$ sind. Dann gilt:
			\begin{itemize}
				\item $f+g$ ist stetig in $x$
				\item $f-g$ ist stetig in $x$
				\item $f\cdot g$ ist stetig in $x$
				\item Falls $g(y)\not = 0, \forall y\in I$, so ist $\frac{f}{g}$ stetig in $x$
			\end{itemize}
		\end{Satz}
	\subsection{Komposition stetig er Funktionen}
		\begin{Satz}
			Seien $I, J$ Intervalle, $f: I\to\R$ und $g: J\to \R$ und $f(I)\subset J$ \newline 
			Ferner sei $f$ stetig in $x\in I$ und $g$ stetig in $y=f(x)$ \newline
			Dann ist $g\circ f: I\to \R$ stetig in $x$
		\end{Satz}
	\subsection{Zwischenwertsatz}
		\begin{Satz}
			Sei $f:[a, b]\to\R$stetig auf dem abgeschlossenen Intervall $[a, b]$. Dann nimmt $f$ in $(a, b)$ jeden beliebigen Wert $y$ zwischen $f(a)$ und $f(b)$ an
		\end{Satz}
		\begin{Satz}
			Sei $f:[a, b]\to \R$ stetig auf dem abgeschlossenen Intervall $[a, b]$. Dann nimmt $f$ in $[a, b]$ jeden beliebigen Wert
			$$
				y\in [\min_{x\in[a, b]}f(x), \max_{x\in[a, b]}f(x)]
			$$
			an
		\end{Satz}
	\subsection{Satz über Nullstellen}
		\begin{Satz}
			Sei $f:[a, b]\to \R$ stetig auf dem abgeschlossenen Intervall $[a, b]$ und es gelte $f(a)<0<f(b)$ oder $f(a) > 0 > f(b)$. Dann hat $f$ in $(a, b)$ mindestens eine Nullstelle, d.h. es existiert ein $x\in(a, b)$ mit $f(x)=0$
		\end{Satz}
	\subsection{Satz von Minimum und Maximum}
		\begin{Satz}
			Sei $f:[a, b]\to \R$ stetig auf dem abgeschlossenen Intervall $[a, b]$. Dann nimmt $f$ in $[a, b]$ Maximum und Minimum an, d.h. es existieren $x_{\min}
			, x_{\max} \in [a, b]$ mit 
			$$
				f(x_{\min})\leq f(x) \leq f(x_{\max}, \quad \forall x\in [a, b]
			$$
		\end{Satz}
		Insbesondere gilt für $x_{\min}$ und $x_{\max}$
		\begin{align*}
			f(x_{\min}) &= \inf_{x\in[a, b]} f(x) = \min_{x\in [a, b]} f(x) \\
			f(x_{\max}) & = \sup_{x\in [a, b]} f(x) = \max_{x \in [a, b]} f(x)
		\end{align*}
		\begin{definition}
			Sei $(x_n)$ eine reelle Folge. Wir schreiben $x_n\to\infty$, wenn gilt
			$$
				\forall C \in \R\exists n_0\in\N\forall n\geq n_0: x_n\geq C
			$$
			Analog schreiben wir $x_n\to -\infty$, wenn gilt
			$$
				\forall C \in\R\exists n_0\in\N\forall n\geq n_0:x_n\leq C
			$$
		\end{definition}
	\subsection{Metrik in normierten Räumen}
		