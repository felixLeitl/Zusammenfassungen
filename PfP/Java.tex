\section{Java}
\subsection{Grundlagen}
\subsubsection{Terminologie}
\begin{itemize}
	\item kritischer Abschnitt
	\item Verklemmung / Deadlock
	\item intrinsic lock (Marke jedes Objekts)
	\item Serialisierung (durch sycnchorized)
	\item Speichermodell
	\item Datensynchonisation: 
		\begin{itemize} 
			\item Koordination der Zugriffe
			\item Kein gemeinsamer Zustand
			\item Unveränderlichke Datenstrukturen
		\end{itemize}
	\item lokale Korrektheit: Eine Klasse C ist lokal korrekt, wenn keine Folge von Operationen zu einem nicht-spezifikationsgemäßen Zustand führen kann
	\item thread-sicher:
		\begin{itemize}
			\item wenn sie lokal korrekt ist und
			\item wenn bei beliebig verschränkten, nebenläufigen Ausführungen ihrer Methoden das Verhalten spezifikationsgemäß bezüglich der Pre- und Post-Bedingungen und der Klasseninvariante bleibt
		\end{itemize}
	\item wechselseitiger Ausschluss
	\item Verhungerung
	\item Daten- / Task-paralleles Vorgehen
	\item Lastbalance
	\item Turnierverfahren: Prozessoren dazu bzw. abschalten, nach Stufe des Turniers
\end{itemize}
\subsubsection{Speedup}
$$
	S_p(n)=\frac{T^*(n)}{T_p(n)}
$$
\begin{itemize}
	\item $T^*(n)$: ist die Laufzeit des schnellsten sequentiellen Algorithmus
für das Problem bei Eingabegröße $n$
	\item $T_p(n)$: ist die Laufzeit des parallelen Programms mit $p$
Prozessoren (bei Eingabegröße $n$)
\end{itemize}
Idealer Speedup bedeutet, dass der Speedup eines
Programms mit der Prozessorzahl übereinstimmt \\
Gesetz von Amdahl:
$$
	S_p(n)=\frac{T^*(n)}{s(n)\cdot T^*(n) + \frac{p(n)}{p}\cdot T^*(n)}
$$
\begin{itemize}
	\item in sequentielle Anteile $s(n)$, die nicht parallelisierbar sind
	\item in parallele Anteile $p(n)$, die auf mehrere Threads verteilt werden können
\end{itemize}
Lemma von Brent: Wenneseinenstatischen Algorithmus mit $T_p(n)=\BigO(t)$ gibt, bei dem alle Prozessoren zusammen $s$ Schritte ausführen, dann gibt es einen statischen Algorithmus mit $T_{s/t}(n)=\BigO(n)$
\subsection{Parallelisierung}
\subsubsection{Threads}
\begin{minted}{Java}
public static class InnerThread extends Thread{
    private final int threadNumber:
    public InnerThread(int threadNumber){
        this.threadNumber = threadNumber;
    }
    public void run(){
        Sytsem.out.println("Thread: " + threadNumber)
    }
}
Thread[] innerThreads = new Thread[10];
for(i=0; i < 10; i++){
    innerThreads[i] = new InnerThread(i);
    innerThreads[i].start();
}
for thread in innerThread{
    try{
        thread.join();
    }catch(InterruptedExeption e){
        System.out.println(e + " has been thrown")
    }
}
\end{minted}
\subsubsection{Runnabels}
\begin{minted}{Java}
public static class Runner implements Runnable{
    private final int threadNumber;
    public InnerThread(int threadNumber){
    this.threadNumber = threadNumber;
    }
    public void run(){
        System.out.println("Thread: " + i);
    }
}
Thread[] threads = new Thread[10];
for(i = 0; i < 10; i++){
    threads[i] = new Thread(new Runner(i))
    thread[i].start();
}
for thread in threads{
    try{
        thread.join();
    }catch(InterruptedExeption e){
        System.out.println(e + " has been thrown")
    }
}
\end{minted}
\subsubsection{ExecutorService}
\begin{minted}{Java}
public static class Executor implements Runnable{
    private final int threadNumber;
    public Executor(int threadNumber){
        this.threadnumber = threadNumber;
    }
}
ExecutorService pool = Executors.newFixedThreadPool(10); // allows 10 parallel Threads
for(i = 0; i < 20; i++){
	pool.execute(new Executor(i));
}
pool.shutDown();
try{
	pool.awaitTermination(60, TimeUnit.SECONDS);
}catch(InterruptedExption e){
	System.out.printl(e + " has been thrown")
}	
\end{minted}
\subsection{Sicherung}
\subsubsection{Synchronized}
Synchronyzed garantiert nur bei Verwendungen der selben Marke, dass die Werte vor der Verwendung der Marke sichtbar werden
\begin{minted}{Java}
public synchronized void doSmth(){} // takes and gives intrinsic lock (this) back (atomic)
synchronized(this){ // takes and gives intrinsic lock (this / Object) back (atomic)
	// Do Something in here
}
\end{minted}
\subsubsection{volatile}
Zur Sicherstellung der Sichtbarkeit und um primitive Type atomar zu machen
\begin{minted}{Java}
private volatile int number; // primitive types are atomic in Java, when used with volatile
\end{minted}
\subsubsection{AtomicInteger}
\begin{minted}{Java}
int decrementAndGet();
int incrementAndGet();
int addAndGet(int delta);
int getAndDecrement();
int getAndIncrement();
int getAndAdd(int delta);

AtomicInteger number = new AtomicInteger(10);
number.incermentAndGet(5);
\end{minted}
\subsubsection{ReentrantLock}
\begin{minted}{Java}
ReentrantLock lock = new ReentrantLock();
lock.lock();
try{
}finally{ // error handling
lock.unlock();
}
\end{minted}
\subsubsection{CyclicBarrier}
\begin{minted}{Java}
CyclickBarrier barrier = new CyclicBarrier(3);
barrier.await();
barrier.await(60, timeUnit.SECONDS) // waits 60s
\end{minted}
\subsubsection{BlockingQueue}
\begin{minted}{Java}
LinkedBlockingQueue<Integer> queue = new LinkedBlockingQueue
queue.put(10);
queue.offer(10); // returns true on success
queue.take(); // takes last element, when empty => blocks, till one element is in queue
queue.poll(); // returns last element or null
\end{minted}
\subsubsection{CountDownLatch}
\begin{minted}{Java}
CountDownLatch countDown = new CountDownLatch(10);
countDown.countDown();
countDown.getCount();
countDown.await() // blocks till countdown.getCount==0
\end{minted}















