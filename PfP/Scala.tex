\section{Scala}
\subsection{Grundlagen}
\subsubsection{Terminologie}
\begin{itemize}
	\item Seiteneffekt
	\item Werttreue: gleiche Eingabe ergibt gleiche Ausgabe
	\item REPL: Read-Eval-Print-Loop
	\item Objektorientiert: In Scala ist alles ein Objekt
	\item statisch typisiert: Typen werden zur Übersetzung festgelegt
	\item implizite Typinferenz
	\item Funktionsabstraktion: Funktionsdefinition
	\item Currying
	\item Paradigmen: imperativ vs. deklarativ
	\item Funktion höherer Ordnung: Parameter oder Rückgabewert sind Funktion
	\item anonyme Funktion: kann einer Variable zugeordnet werden
	\item n-stellige Funktion: n Parameterlisten
	\item Lazy Evaluation: Berechnung erst bei Verwendung
	\item LazyList: Unendlichkeit möglich
	\item Overhead: Mehrkosten durch Parallelisierung
\end{itemize}
\subsection{Syntax}
\subsubsection{List}
\begin{minted}{scala}
List(4,5,6): List[Int]
List(42, 'a', false): List[Any] // mehrere Typen in Liste => Supertyp Any
List.range(4,7) // 7 exclusiv
4 to 6
4::5::6::Nil // :: adds or removes the first element
List(4,5):::List(6) // adds two Lists together
List().length
List().head // first element
List().tail // List, without head
List().drop(3) // drops the first n elements
List().sum // adds all elements
List().product
List(4,5,6)(0) // returns first element
List(4,5,6).take(2) // returns first n elements
List().reverse
List(1,2,3).zip(List(4,5,6,7)) // => List((1,4),(2,5),(3,6)) drops the tail
ls.foldRight(Nil)((e, r) => if (r == Nil || e != r.head) e :: r else r)
List(1,2,3).filter(_ == 3) // returns with only 3
List(List(2), List(4,3)).flatten // returns List(2,3,4)
List(1,2,3).equals(List(1,2,3))
List().map(_ + 1) // increases every Element by one
"Scala".toList
List(1,2,3,4,5).takeWhile(_ < 3) // returns List(1,2)
List().dropWhile(_ < 5)
List('S', 'c', 'a').mkString
List(2,4,6).forall(_ % 2 == 0) // returns true, when all are true
List(2,3,4).exists(_ == 3) // returns true, when on is true
\end{minted}
\subsubsection{LazyList}
\begin{minted}{scala}
1#::LazyList()
lazy val dikrim = b*b - 4  * a * c // calculation at first use
LazyList.iterate(BigInt(2))(_ + 1) // LazyList form 2 to infinity
\end{minted}
\subsubsection{Tupel}
\begin{minted}{scala}
(false, 10): (Boolean, Int)
(false, 10, 'a'): (Boolean, Int, Char)
(false, 10)._1 // gives you the first element
\end{minted}
\subsubsection{Funktionen}
\begin{minted}{scala}
def foo: Boolean => Char = a => if a 'a' else 'b'
def pair[A]: A => (A,A) = x => (x, x) // Generische Typparameter
def addd: Int => Int => Int => Int = x => y => z => x + y + z // Currying
def bar: List[Int] => Int = { // Mustervergleich
  case a::as => 1 + bar(as)
  case a => 0
}
def lol: Unit => Int = _ => 42 // Unit is nothing and _ ignores the constant
val f = (x: Int) => x + 1 // anonymus function
val f = _ + 1 // anonymus function
\end{minted}
\subsubsection{If-Else}
\begin{minted}{scala}
if n>2 then n else -n
if (n>2) n else -n
if n>2 then n
else if n==2 then 2
else -n
case x if x > 2 // guard
\end{minted}
\subsubsection{For-comprehension}
\begin{minted}{scala}
for x <- List.range(1,10) yield x*x
for x <- List.range(1,10) if x > 1 yield x*x
for x <- List.range(1,5)
    y <- List.range(x, 4) yield (x,y)
 for xs <- xss; x <- xs yield x // flatten an list
\end{minted}
\subsubsection{Type Aliasing}
\begin{minted}{scala}
type Coordinate = (Int, Int)
type Move = Coordinate => Coordinate // function typ
def left: Move = {case (x, y) => (x-1, y)}
type Pair[T] = (T,T)
def mult: Pair[Int] => Int = (x, y) => x * y
def copy[T]: Pair[T] => T = x => (x, x) // generic typing
\end{minted}
\subsubsection{Typverbund}
\begin{minted}{scala}
trait Shape // interface
case class Circle(r: Double) extends Shape // constructor is r: Double
case class Rect(l: Double, w: Double) extends Shape
case object Dot extends Shape // Singlton
def square: Double => Shape =
    n => Rect(n, n)
def area: Shape => Double ={
    case Circle(r) => math.Pi * r * r
    case Rect(l, w) => w * l
\end{minted}
\subsubsection{Enum}
\begin{minted}{scala}
enum Answer:
	case Yes, No, Unknown // allows Yes instead of Answer.Yes
import Answer._ 
def flip: Answer => Answer = {
    case Yes => No
    case No => Yes
    case Unknown => Unknwon
\end{minted}
\subsubsection{MapReduce}
\begin{minted}{scala}
val Letters: List[Char] = List.range('a', ('z' + 1).toChar)
def mapLetterCount: String => List[(Char, Int)] = 
    string => count(letters, string.toList)
def reduceLetterCount: ((Char, List[Int])) => (Char, Int) = {
    case (key, values) => (key, values.sum)
}
mapReduce(mapLetterCount)(reduceLetterCount)(bsp1) // maps first, then reduces	
\end{minted}
\subsection{Parallelisierung}
\subsubsection{Vorteeile}
Alle parallelisierbaren Operationen arbeiten wegen \glqq{}divide and conquer\grqq{} automatisch parallel \\
Programmierende muss sich nicht um Lastverteilung/Synchronisierung kümmern \\
Wird im Hintergrund mit einem schlafenden fork/join-Pool umgesetzt
\subsubsection{.par}
\begin{itemize}
	\item Liefert Sicht auf Ursprüngliche Daten
	\item Die Verarbeitungsoperationen der Sicht sind jedoch parallelisiert
	\item Benötigt \verb|import scala.collections.parallel.CollectionConverters._|
	\item Beispiele: \verb|List(1,2,3).par.map(_ + 1)|
\end{itemize}
\subsubsection{Future}
\begin{itemize}
	\item Einpacken einer Berechnung \verb|f| in ein Futur: \verb|val c = Future {f}|
	\item Warten auf das Ergebnis: \verb|Await.result(c, Duration.Inf)|
\end{itemize}
\subsubsection{Akka-Aktoren}
\begin{itemize}
	\item externe Bibliothek
\end{itemize}
\subsection{Datentypen}
\begin{itemize}
	\item Boolean
	\item Int
	\item String
	\item Char
	\item BigInt
	\item BigDecimal
	\item Long
	\item Any: Übertyp aller Typen
	\item Nothing: Untertyp aller Typen
\end{itemize}
\subsubsection{ParVector}
Ist die parallele Version der Liste
\begin{minted}{scala}
342 +: 42 +: ParVector.fill(10)(100)
\end{minted}
\subsubsection{ParRange}
\begin{minted}{scala}
ParRange(30, 47, 1, true) // == (30 to 47)
\end{minted}
\subsubsection{Future}
\begin{minted}{scala}
val a = Future(x*x)
Await.result(a, Duration.Inf)
\end{minted}





