\section{Bewusstsein}
\subsection{Bewusstsein und Bewusstseinsveränderungen}
\subsubsection{Zwei Bedeutungen von \glqq{}Bewusstsein\grqq{}}
\begin{itemize}
	\item Explizierbare Inhalte des inneren Erlebens und Erkennens
		\begin{itemize}
			\item Bewusstseinsstrom: Gesamtheit im \glqq{}Hier und Jetzt\grqq{} erlebter eigener Zustände und Aktivitäten wie Wahrnehmungen, Gedanken, Gefühle, Vorstellungen, Wünsche, Handlungen etc.
			\item Bewusstseinsinhalte sind deklarativ / explizit
		\end{itemize}
	\item Allgemeiner Geisteszustand
		\begin{itemize}
			\item Schlaf vs. Wachzustand
			\item bewusst vs. bewusstlos sein
			\item Veränderung des Bewusstseins durch Hirnverletzungen, Medikamente/Drogen, psychische Krankheiten etc.
		\end{itemize}	
\end{itemize}
\subsubsection{Un-/vorbewusste Inhalte}
\begin{itemize}
	\item Unbewusste (implizite) Prozesse
		\begin{itemize}
			\item Informationen die weder dem Bewusstsein noch dem Gedächtnis zugänglich sind
			\item z.B. Blutdruck und Puls, sensorische Verarbeitung in der Retina
		\end{itemize}
	\item Vorbewusste Prozesse
		\begin{itemize}
			\item Inhalte, die meistens nicht bewusst sind, aber ins Bewusstsein geholt werden können
			\item z.B. Atmung, explizite Gedächtnisinhalte (Erinnerung an den Vornamen)
		\end{itemize}
\end{itemize}
\subsubsection{Funktionen des Bewusstseins}
\begin{itemize}
	\item Selektionsvorteil bewusster kognitiver Funktionen
		\begin{itemize}
			\item Reduktion und Auswahl des Stroms an Reizen \rightarrow Aufmerksamkeitsfunktionen 
			\item Selektive Speicherung \& Aufrechterhaltung von relevanten Informationen \rightarrow Funktion des Arbeitsgedächtnisses
			\item Komplexere explizite Handlungsplanungnen \rightarrow Exekutive Kontrollfunktionen
		\end{itemize}
	\item Konstruktion einer persönlichen und sozialen Realität
		\begin{itemize}
			\item Konstruktion eines stabilen expliziten Selbstkonzepts
			\item Deklaration bewusster Inhalte ermöglichen soziale Kommunikation über Symbole (z.B. Sprache) \& Kooperation
		\end{itemize}
\end{itemize}
\subsubsection{Nachweiß unbewusster Prozesse}
\begin{itemize}
	\item Um einen unbewussten Prozess nachzuweisen, müssen Forschende:
		\begin{itemize}
			\item ein geeignetes nicht-deklaratives Maß entwickeln, dass die fraglichen Prozesse misst und
			\item ein deklaratives Maß, von dem angenommen wird, dass es einen verwandten (aber unabhängigen) bewussten Prozess erfasst und zeigen, dass
			\item eine experimentelle Intervention einen differenziellen Effekt auf das nicht-deklarative und das deklarative Maß hat
		\end{itemize}
\end{itemize}
\rightarrow Experimenteller Nachweiß der Dissoziation von unbewussten und bewussten Prozessen
\subsubsection{Implizites (unbewusstes) Gedächtnis}
\begin{itemize}
	\item Beispiel für implizites Gedächtnis: Priming
		\begin{itemize}
			\item Präsentation eines Reizes beeinflusst unbewusst die Verarbeitung und Reaktion auf wiederholte Präsentation desselben (oder sehr ähnlichen) Reizes
		\end{itemize}
\end{itemize}
\subsubsection{Drei Arten der unbewussten Emotion}
\begin{enumerate}
	\item Verdrängung
		\begin{itemize}
			\item Emotionserzeugender Stimulus wird bewusst wahrgenommen, aber die emotionale Reaktion ist unbewusst
		\end{itemize}
	\item Emotionale Intuition
		\begin{itemize}
			\item Emotionserzeugender Stimulus wird nicht bewusst wahrgenommen, aber die emotionale Reaktion wird bewusst
		\end{itemize}
	\item Vollständige Ahnungslosigkeit
		\begin{itemize}
			\item Weder der Emotionserzeugende Stimulus noch die emotionale Reaktion werden bewusst
		\end{itemize}
\end{enumerate}
\subsection{Veränderte Bewusstseinszustände}
\subsubsection{Hirnaktivität im Wach- und Schlafzustand}
EEG-Muster im REM (rapid eye movement) wie bei Wachheit \rightarrow Träume.

Schlaf wir intern durch circadianen Rhythmus, extern durch Reize und Verhalten gesteuert
\subsubsection{Wozu schlafen wir}
\begin{itemize}
	\item Gedächtniskonsolidierung
		\begin{itemize}
			\item Schlaf hilft v.a. deklarative Gedächtnisinhalte
					dauerhaft zu speichern: Rolle des Hippokampus
		\end{itemize}
	\item Energiekonservierung
		\begin{itemize}
			\item Wer schläft, verbraucht weniger Energie (siehe auch Winterschlaf)
		\end{itemize}
	\item Regenerierung
		\begin{itemize}
			\item Aufbau- und Heilungsprozesse geschehen vorwiegend im Schlaf
			\item Z.B. Steigerung der Transmitter- und Hormonproduktion und Entfernung neurotoxischer Abfallprodukte des neuronalen Zellstoffwechsel
		\end{itemize}
\end{itemize}
\subsubsection{Verändertes Bewusstsein durch Drogen}
\begin{itemize}
	\item Psychoaktive Substanzen
		\begin{itemize}
			\item Binden an spezifischen Synapsen Rezeptoren
			\item Kurzfristig Veränderung des Bewusstseins
			\item Langfristig
				\begin{itemize}
					\item Körperliche Abhängigkeit
					\item Psychische Abhängigkeit
					\item Körperliche und soziale Folgen
				\end{itemize} 
		\end{itemize}
\end{itemize}








