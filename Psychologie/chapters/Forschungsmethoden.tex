\section{Forschungsmethoden}
\subsection{Forschungsprozess}
\subsubsection{Prototypischer Forschungszyklus}
\begin{enumerate}
  \item Theorie
  \item Hypothese
  \item Operationalisierung
  \item Analyse
  \item Publikation
  \item Diskussion \& Lösung offener Fragen
\end{enumerate}
Nach der Analyse wird die Hypothese angepasst, bis diese bereit ist veröffentlicht zu werden.
\subsection{Untersuchungsdesign}
\subsubsection{Korrelationsstudien}
\begin{itemize}
	\item Es wird keine der untersuchten Variablen experimentell manipuliert 
	 
		\rightarrow keine kausalen Schlüsse möglich 
	\item Es werden alle Merkmale so gemessen, wie sie in der Stichprobe angetroffen werden
	\item z.B. Epidemiologische Studien, Umfragen, Mehrzahl der Studien in der Persönilichkeitspsychologie
	\item Beobachtung des Zusammenhangs von natürlich auftretenden Merkmalen
	\item Kausalität kann nicht allein aus der Korrelation zweier Variablen abgeleitet werden (Kausaliätsproblem)
	\item Zusammenhang zwischen zwei Variablen ist manchmal nur scheinbar (Problem der dritten Variablen)
	\item Korrelative Zusammenhänge können keine Interventionen begründen
\end{itemize}
\subsubsection{Experimentelle Studien}
In Experimenten wird ein/mehrere Merkmale experimentell manipuliert und die Auswirkung dieser auf andere Variablen gemessen
\begin{itemize}
	\item Manipuliert: Unabhängige Variable (UV)
	\item Gemessen: Abhängige Variable (AV)
	\item Between-subject vs. within-subject Design  
\end{itemize} 
z.B. Mehrzahl der Studien aus Sozial-, Kognitions- und Biopsychologie \newline \newline
Hauptmerkmale (Between-subject):
\begin{itemize}
	\item Randomisierung \rightarrow Kontrolle externer Einflüsse
	\item Manipulation der unabhänigen Variablen
	\item Messung der abhängigen Variablen
\end{itemize}
$p$: Wahrscheinlichkeit, dass der Effekt zufällig zustande gekommen ist 

$p < 0.05$ wird als \glqq{}signifikant\grqq{} betrachtet \newline \newline

Hauptmerkmale (Between-subject):
\begin{itemize}
	\item Randomisierte Manipulation der unabhängigen Variablen
	\item Mehrfache Messung der abhängigen Variablen
\end{itemize}
Vorteile von Experimentalstudien: 
\begin{itemize}
	\item Kausalzusammenhänge lassen sich ableiten
\end{itemize}
Nachteil von Experimentalstudien:
\begin{itemize}
	\item Manche Merkmale lassen sich nicht oder nicht leicht unter
experimentelle Kontrolle bringen
\end{itemize} 
\subsection{Messungen in der Psychologie}
\subsubsection{Deklarative Messverfahren}
\begin{itemize}
	\item Selbstbericht
	\item Fragebögen
	\item Interviews
	\item Wahrnehmungsurteil
\end{itemize}
\subsubsection{Nicht-deklarative Messverfahren}
\begin{itemize}
	\item Inhaltsanalyse
	\item Kognitive Tests
	\item Verhaltenstests
	\item Physiologische Messungen
\end{itemize}
\subsubsection{Hauptgütekriterien von Messungen}
\begin{itemize}
	\item Objektivität (Ausmaß, in dem ein Test frei von subjektiven Einflüssen des/der
VersuchsleiterIn ist)
	\item Reliabilität (Ausmaß, in dem ein Test bei wiederholter Anwendung ähnliche
Ergebnisse liefert)
	\item Validität (Ausmaß, in dem ein Test das psychologische \glqq{}Konstrukt\grqq{} misst, das
er zu messen vorgibt)
\end{itemize}





















