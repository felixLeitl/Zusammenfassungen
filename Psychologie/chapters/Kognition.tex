\section{Kognition}
\begin{itemize}
	\item Allgemeinpsychologische Prozesse der Informationsgewinnung, -verarbeitung, und – nutzung, einschließlich Wahrnehmung, Aufmerksamkeit, Gedächtnis, Wissen, Sprache, Entscheiden, Problemlösen, logisches Denken etc.
	\item Kognitive Prozesse sind nicht direkt beobachtbar, sondern müssen aus Verhalten bzw. Gehirnaktivität erschlossen werden!
	\item Kognition $\not=$ Bewusstsein, da viele kognitive Prozess implizit sind
	\item Intelligenz: Individuelle Fähigkeit, mit seinen kognitiven Prozessen Probleme zu lösen  
\end{itemize}
\subsection{Methodischer Ansatz}
Vermessung kognitiver Prozesse durch Vergleich von Reaktionszeit. z.B. Farben und Nabe der Farbe trennen.

Der Reaktionszeitunterschied zwischen den automatischen (impliziten) Prozessen und Kontrollierten (expliziten) Prozessen kann auf kognitive Prozess schließen
\subsection{Sprachverständnis und Sprachproduktion}
Sprache: System der Kommunikation, dass Laute (Phoneme) und Schriftzeichen (Grapheme) nach grammatikalischen Regeln (Syntax) benutzt, um Gedanken, Erfahrungen, Ideen und Gefühle etc. (Bedeutung/Semantik) durch Wörter (Lexikon) auszutauschen
\subsubsection{Zentrale kognitive Prozesse der Sprache}
\begin{itemize}
	\item Sprachproduktion: Kognitive Prozesse, durch die wir Bedeutung symbolisch, z.B. durch Sprachlaute, ausdrücken
		\begin{itemize}
			\item Übersetzung von Gedanken (Bedeutung/Semantik) in Wörter (Lexikon) und Sätze nach grammatikalischen Regeln (Syntax) mit bestimmten Lauten (Phoneme)
			\item Hörerbezug: Abstimmung der Äußerung auf die Hörerschaft, für die sie gedacht ist (Pragmatik).
		\end{itemize}
	\item Sprachverständnis: Kognitive Prozesse, durch die wir die Bedeutung von sprachlichen Äußerungen erfassen
		\begin{itemize}
			\item Laute werden als Wörter wahrgenommen, deren Bedeutung zusammen mit der Syntax die Bedeutung des Satzes ergibt
			\item Schwierigkeit: Lexikalische und syntaktische Mehrdeutigkeit
		\end{itemize}
\end{itemize}
\subsubsection{Kognitive Prozesse beim Sprechen}
\begin{enumerate}
	\item Semantische Ebene
	\item Lexikalische Ebene
	\item Syntaktische Ebene
	\item Phonologische Ebene: Wegstabenverbuchseln
\end{enumerate}
\subsubsection{Mehrdeutigkeit beim Sprachverständnis}
Reihenfolge des Sprachverständnisses umgekehrt zum Sprechen
\subsection{Problemlösen und schlussfolgerndes Denken}
\begin{itemize}
	\item Induktives Schlussfolgern:
		\begin{itemize}
			\item Generalisierendes Schlussfolgern aus einzelnen Beobachtungen 
		\end{itemize}
	\item Deduktives Schlussfolgern:
		\begin{itemize}
			\item Spezifische Schlussfolgerung aus einer Prämisse/Hypothese
		\end{itemize}
\end{itemize}
\subsubsection{Induktives Schlussfolgern: Kognitive Heuristiken und Verzerrungen}
Induktive Schlüsse im Alltag:
\begin{itemize}
	\item Verfügbarkeitsheuristik: Ereignisse, an die wir uns leichter erinnern können, halten wir für wahrscheinlicher
	\item Repräsentativitätsheuristik: Objekte/Ereignisse/ Personen werden wahrscheinlicher einer Kategorie zugeordnet, wenn sie repräsentativ für die Kategorie erscheinen
	\item Bestätigungsverzerrung: Menschen suchen vor allem bestätigende Information und nehmen diese selektiv wahr!
\end{itemize}
\subsection{Intelligenz und Intelligenzdiagnstik}
\subsubsection{Psychometrische Intelligenztheorien}
\begin{enumerate}
	\item g-Faktorentheorie: Ein übergeordneter g-Faktor (Allgemeine Intelligenz) erklärt gut die Leistungen über verschiedene kognitive Fähigkeitstests hinweg
	\item Zweifaktorentheorie: Kristalline und Fluide Intelligenz
	\item Dreischichtenmodell: Vereint und erweitert I. \& II.
\end{enumerate}
\subsubsection{Intelligenzdiagnostik}
ProbandIn bearbeitet Aufgaben, die spezifische kognitive
Fähigkeiten erfassen: Schnelligkeit vs. Niveau. Vergleich mit Normstichprobe im gleichen Alter \rightarrow IQ






