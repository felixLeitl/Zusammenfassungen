\section{Lernen}
Lernen: Ein erfahrungsbasierter Prozess, der in einer überdauernden Änderung des Verhaltens oder des Verhaltenspotentials resultiert.

Verhaltensanalyse: Analyse der früheren Lerngeschichte, um gegenwärtiges Verhalten zu erklären (und möglicherweise zu ändern).

\subsection{Behaviorismus}
Verhalten kann alleine durch Lernprozesse verstanden werden.

Die Untersuchung von Erleben ist unwissenschaftlich, da nicht von außen beobachtbar: Introspektion und Selbstberichte werden abgelehnt, nur Verhalten kann objektiv beobachtet und beschrieben werden.

Radikaler Behaviorismus lehnt alle biologischem Prozesse und Präpositionen ab und ist wissenschaftlich widerlegt, jedoch nicht alle Erkenntnisse des Behaviorismus (v.a. Lernen) 

\subsection{Klassische Konditionierung}
Eine Art des Lernens, bei der das erlernte Verhalten (konditionierte Reaktion, CR) durch einen Stimulus (konditionierter Stimulus, CS) hervorgerufen wird, der seine Wirkung durch eine Assoziation mit einem biologisch (=emotional, motivational) bedeutsamen Stimulus (unkonditionierter Stimulus, UCS) erlangte.

\begin{itemize}
	\item UCS \rightarrow Unkonditionierter Stimulus
	\item UCR \rightarrow Unkonditionierte Reaktion
	\item NS \rightarrow Neutraler Stimulus
	\item CS \rightarrow Konditionierter Stimulus
	\item CR \rightarrow Konditionierte Reaktion
\end{itemize}

\subsubsection{Reizgeneralisierung vs. Reizdiskriminierung}
Reizgeneralisierung: CR auf CS ähnliche Stimuli/Reize, die selbst nie mit dem UCS gepaart wurden

Reizdiskrimierung: Ein Konditionierungsprozess, bei dem der Organismus lernt, nicht auf Reize zu reagieren, die sich vom CS entlang einer Dimension unterscheiden.

Bsp. für KK:
\begin{itemize}
	\item Smartphones
	\item Spezifische Phobien
	\item Werbung
	\item Kontext -> z.B. Hunger, Schlaf 
 	\item Esspräferenzen
 	\item Placeboeffekt
\end{itemize}

\subsection{Operante Konditionierung}
Eine Lernform, bei der sich die Wahrscheinlichkeit eines Verhalten aufgrund einer Veränderung ihrer Konsequenzen verändert (d.h. erhöht oder verringert)

Auslösender Stimulus hier irrelevant bzw. nur diskriminativer Hinweisreiz.

Die Wahrscheinlichkeit eines Verhaltens ändert sich je nach Art der kontingenten Konsequenz des
Verhaltens
\begin{itemize}
	\item Verstärkung: Erhöht die Wahrscheinlichkeit des Verhaltens
		\begin{itemize}
			\item Positiv: durch Präsentation eines angenehmen/belohnenden Stimulus: Positive Verstärkung
			\item Negativ: durch Entfernung eines angenehmen Stimulus: Bestrafung 1. Art
		\end{itemize}
	\item Bestrafung: Verringert die Wahrscheinlichkeit eines Verhaltens
		\begin{itemize}
			\item I: durch Präsentation eines aversiven Stimulus: Bestrafung 2. Art 
			\item II: durch Entfernung eines angenehmen Stimulus: Negative Verstärkung
		\end{itemize}
\end{itemize}
\subsection{Einschränkungen der Konditionierung}
\begin{itemize}
	\item Einige der Konditionierungen werden viel schneller erworben, weil sie biologisch-evolutionär \glqq{}prepared\grqq{} sind:
	\item KK: Geschmacksaversion
		\begin{itemize}
			\item Schmerzvermeidungsreaktion durch elektrische Schocks gut an visuelle und auditorische Reize, nicht aber an Geschmacksreize koppelbar. Chemisch ausgelöste Übelkeit gut an Geschmack koppelbar (Geschmacksaversion), nicht aber an visuelle und auditorische Reize.
			\item Manche CS-CR Paarungen passen \glqq{}natürlicher\grqq{} zusammen als andere – biologische \glqq{}prepardness\grqq{}
		\end{itemize}
	\item OK: Instinktverschiebung
		\begin{itemize}
			\item Waschbären lernen die Benutzung von Münzen durch operante Konditionierung…und fangen an die Münzen zu rubbeln und zu waschen \rightarrow\ Wasch-Instinkt auf sekundären Verstärker verschoben
			\item Spezifische Phobien auf Spinnen, Schlangen, Höhe, Gewitter \dots\ aber nicht Auto
		\end{itemize}
\end{itemize}
Menschen und Tiere lernen von Anderen \rightarrow\ Kinder lernen von Modell aggressives, wie prosoziales Verhalten. 





