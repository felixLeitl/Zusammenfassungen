\section{Wahrnehmung}
\subsection{Wahrnehmungsprozess}
Kein passiv abbildender, sondern aktiver, konstruierender Vorgang!
\begin{enumerate}
	\item Distaler Stimulus
	\item Transformation Licht
		\begin{enumerate}
			\item Fokussieren auf Retina
			\item Proximaler Stimulus
		\end{enumerate}
	\item Sensorische Prozesse am Rezeptor
		\begin{enumerate}
			\item Transduktion
			\item Neuronale Repräsentation 
		\end{enumerate}
	\item Neuronale Verarbeitung
	\item Wahrnehmung
	\item Erkennen
	\item Handlen
\end{enumerate}
\subsection{Sehen}
\subsubsection{Physik}
Spezifische Wellenlängen (380 - 780 nm) werden als verschiedene Farben wahrgenommen.

\subsubsection{Fokussieren auf der Retina}
Cornea und Linse (80/20\% fix/variabel) fokussieren Bild auf Retina (Netzhaut). Visuelle Rezeptoren auf der Netzhaut (Stäbchen und Zapfen) enthalten visuelle Pigmente \rightarrow Transduktion. Sehnerv leitet Informationen von Retina ans Gehirn weiter.

\subsubsection{Transduktion im Visuellen Rezeptor}
Durch auftreffen von Licht auf den visuellen Rezeptor gehen die Na$^+$ Kanäle zu \Rightarrow Zelle wird hyperpolarisiert und schüttet weniger Glutamat aus

\subsubsection{Neuronale Verarbeitung}
\begin{center}
	\includegraphics[scale=.2]{img/Auge.png}
\end{center}

\subsubsection{Verarbeitung Wellenlängen}
Zapfen:
\begin{itemize}
	\item S-Cone (Blau)
	\item M-Cone (Grün)
	\item L-Cone (Gelb)
\end{itemize}

\rightarrow Trichromatische Theorie

Gegenspieler-Verschaltung:
\begin{itemize}
	\item Rot-Grün (L-M)
	\item Blau-Gelb (S-(M+L))
	\item Schwarz-Weiß (S+M+L)
\end{itemize}

\rightarrow Gegenfarbtheorie
\subsection{Hören}
\subsubsection{Transformation \& Transduktion}
%TODO: Hören
